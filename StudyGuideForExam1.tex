\documentclass[12pt]{article}
\usepackage{color,latexsym,fancyhdr,amsmath,amsfonts,dsfont,amssymb}
\usepackage{color,soul}
\newtheorem{theorem}{Theorem}[section]


\newtheorem{claim}[theorem]{Claim}
\newcommand{\Z}{\mathds{Z}}
\newcommand{\R}{\mathds{R}}
\newcommand{\B}{\mathcal{B}}
\newcommand{\T}{\mathcal{T}}


\topmargin        -0.2 in
\textheight       8.4 in
\oddsidemargin    0 in  
\evensidemargin   0 in     
\textwidth        6.5 in
\headheight       15pt     
\headsep          .35 in     


\begin{document}
\pagestyle{fancy} \lhead{MTH 415 Study Guide} \chead{Exam 1}
\rhead{Austin Klum} 
\lfoot{} \cfoot{} \rfoot{}

\setcounter{section}{-1}

\section{}
	\subsection{DeMorgan's Laws}
		\begin{enumerate}
			\item $ A-(B\cup C) = (A-B)\cap(A-C) $
			\item $ A-(B\cap C) = (A-B)\cup(A-C) $
		\end{enumerate}
	
	\section{}
	\subsection{Definition of a Topology}
		Let $ X $ be a set. A \textbf{topology} $ \T $ on $ X $ is a collection of subsets of $ X $, each called an \textbf{open set} such that:\\
		\\
			(i) $\varnothing$ and $X$ are open sets;\\
			(ii) The intersection of finitely many open sets is an open set;\\
			(iii) The union of any collection of open sets is an open set.\\
		\\
		The set $X$ together with a topology $\mathcal { T }$ on $X$ is called a topological space.
	\subsection{Trivial Topology}
		Define $ \T = \{\varnothing,X\} $. Notice, $ \T $ satisfies all three conditions for being a topology. For obvious reasons, it is called the \textbf{Trivial Topology} define on $ X $.
	\subsection{Discrete Topology}
		Let $ X $ be a nonempty set and let $ \T  $ be the collection of all subsets of $ X $. This is called the \textbf{discrete topology} on $ X $. This is the largest topology that we can define on $ X $.
	\subsection{Finite Complement Topology}
		On the real line, $\mathbb { R } ,$ define a topology whose open sets are the empty set and every set in $\mathbb { R }$ with a finite complement. We call this topology the \textbf{finite complement topology} on $ \R $ and denote it by $ \R_{fc} $
	\subsection{Definition of a Neighborhood}
		Let $X$ be a topological space and $x \in X .$ An open set $U$ containing $x$ is said to be a \textbf{neighborhood} of $x .$
	\subsection{Theorem for using Neighborhood to determine if sets are open}
	Let $X$ be a topological space and let $A$ be a subset of $X .$ Then $A$ is open in $X$ if and only if for each $x \in A ,$ there is a neighborhood $U$ of $x$ such that $x \in U \subset A$
	\subsection{Definition of Basis for a Topology}
		Let $X$ be a set and $\mathcal { B }$ be a collection of subsets of $X .$ We
		say $\mathcal { B }$ is a \textbf{basis (for a topology on $X$)} if the following statements hold:\\
			\\
			(i) For each $x$ in $X ,$ there is $a B$ in $\mathcal { B }$ such that $x \in B$\\
			(ii) If $B _ { 1 }$ and $B _ { 2 }$ are in $\mathcal { B }$ and $x \in B _ { 1 } \cap B _ { 2 } ,$ then there exists $B _ { 3 }$ in $\mathcal { B }$ such that $x \in B _ { 3 } \subset B _ { 1 } \cap B _ { 2 }$. \\
			\\
			We call the sets in $B$\textbf{ basis elements}.
	\subsection{Definition of a Topology Generated by a Basis}
		Let $\mathcal { B }$ be a basis on a set $X .$ The \textbf{topology $\mathcal { T }$generated by} $\mathcal { B }$ is obtained by defining the open sets to be the empty set and every set that is
		equal to a union of basis elements.
	\subsection{Standard Topology}
		On the real line $\mathbb { R } ,$ let $\mathcal { B } = \{ ( a , b ) \subset \mathbb { R } | a < b \}$ The topology generated by $\mathcal { B }$ is called the\textbf{ standard topology} on $\mathbb { R }$\\
		\\
		Open sets in the standard topology on $\mathbb { R }$ are unions of open intervals.
	\subsection{Theorem: Bases Generate Topologies}
	The topology $\mathcal { T }$ generated by a basis $\mathcal { B }$ is a topology.
	\subsection{Lower/Upper Limit Topology}
		On $\mathbb { R } ,$ let $B = \{ [ a , b ) \subset \mathbb { R } | a < b \} .$ The collection $\mathcal { B }$ is a basis for a topology on $\mathbb { R }$ We call the topology generated by this basis the \textbf{lower limit topology} since each basis element contains its lower limit. We denote $\mathbb { R }$ with this topology by $\mathbb { R } _ { l }$ .\\
		\\
		We can similarly define the \textbf{upper limit topology} on $\mathbb { R }$ via the basis $\mathcal { B } =$ $\{ ( a , b ] \subset \mathbb { R } | a < b \}$
	\subsection{Digital Line Topology}
		For each $n \in \mathbb { Z } ,$ define\\
			\[B ( n ) = \left\{ \begin{array} { l l } { \{ n \} } & { \text { if } n \text { is odd } } \\ { \{ n - 1 , n , n + 1 \} } & { \text { if } n \text { is even } } \end{array} \right.\]
		The collection $\mathcal { B } = \{ B ( n ) | n \in \mathbb { Z } \}$ is a basis for a topology on $\mathbb { Z } .$ The resulting topology is called the \textbf{digital line topology}
	\subsection{Theorem: A set is open if and only if every element is contained in a basis element}
		Let $X$ be a set and $\mathcal { B }$ be a basis for a topology on $X .$ Then $U$ is open in the topology generated by $\mathcal { B }$ if and only if for each $x \in U$ there
		exists a basis element $B _ { x } \in \mathcal { B }$ such that $x \in B _ { x } \subset U .$
	\subsection{Open Balls and Standard Topology on $ \R^2 $}
		For each $x$ in $\mathbb { R } ^ { 2 }$ and $\varepsilon > 0 ,$ define
		\[B ( x , \varepsilon ) = \left\{ p \in \mathbb { R } ^ { 2 } | d ( x , p ) < \varepsilon \right\}\]
		The set $B ( x , \varepsilon )$ is called the \textbf{open ball of radius $\varepsilon$ centered at $x$ }. Let
		\[\mathcal { B } = \{ B ( x , \varepsilon ) | x \in \mathbb { R } ^ { 2 } , \varepsilon > 0 \}\]
		So $\mathcal { B }$ is the collection of all open balls associated with the Euclidean distance $ d $. We call the topology generated by $\mathcal { B }$ the \textbf{standard topology on} $\mathbb { R } ^ { 2 }$ .
	\subsection{Theorem: The collection $\mathcal { B } = \{ B ( x , \varepsilon ) | x \in \mathbb { R } ^ { 2 } , \varepsilon > 0 \}$ is a basis for a topology on $ \R^2 $ }
	\subsection{Theorem: If all sets have an open set with element, we have a basis}
		Let $X$ be a set with topology $\mathcal { T } ,$ and let $\mathcal { C }$be a collection
		of open sets in $X .$ If, for each open set $U$ in $X$ and for each $x \in U ,$ there is
		an open set $V$ in $\mathcal { C }$ such that $x \in V \subset U ,$ then $\mathcal { C }$ is a basis that generates the
		topology $T .$
	\subsection{Definition: Closed Sets}
		A subset A of a topological space $X$ is \textbf{closed} if the set $ X-A $ is open. (i.e. if a set is open then it's complement is closed)
	\subsection{Closed Ball}
		For each $x$ in $\mathbb { R } ^ { 2 }$ and $\varepsilon > 0 ,$ define the \textbf{closed ball of radius $\varepsilon$ centered} at $x$ to be the set
			\[\overline { B } ( x , \varepsilon ) = \left\{ y \in \mathbb { R } ^ { 2 } | d ( x , y ) \leq \varepsilon \right\}\]
		where $d ( x , y )$ is the Euclidean distance between $x$ and $y$\\
		\\
		If $[ a , b ]$ and $[ c , d ]$ are closed bounded intervals in $\mathbb { R } ,$ then the
		product $[ a , b ] \times [ c , d ] \subset \mathbb { R } ^ { 2 }$ is called a \textbf{closed rectangle}.
	\subsection{Theorem: Closed balls and closed rectangles are closed sets in the standard topology on $ \R^2 $ }
	\subsection{Warning: A set can be open, closed, both, or neither. If a set is not opened $ \not \Rightarrow$ a set is closed }
	\subsection{Theorem: Closed Sets have the same properties of open sets}
		Let $X$ be $a$ topological space. The following statements about the collection of closed sets in $X$ hold:\\
		\\
		(i) $\varnothing$ and $X$ are closed.\\
		(ii) The intersection of any collection of closed sets is a closed set.\\
		(iii) The union of finitely many closed sets is a closed set.\\
	\subsection{Definition: Hausdorff}
		A topological space $X$ is \textbf{Hausdorff} if for every pair of distinct points$x$ and $y$ in $X ,$ there exist disjoint neighborhoods $U$ and $V$ of $x$
		and $y ,$ respectively. \\
		\\
		Points are "housed off" from other points by disjoint neighborhoods.
	\subsection{Theorem: If $X$ is a Hausdorff space, then every single-point subset of $X$ is closed.}

\section{}

	\subsection{Interior}
	Let $ A $ be a subset of a topological space $X .$ The\textbf{ interior of }$ A $, denoted $\dot { A }$ or Int(A), is the union of all open sets contained in $A$.\\
	\\
	The interior of $ A $ is open and a subset of $ A $. 
	\subsection{Closure} 
		Let $ A $ be a subset of a topological space $X .$  The \textbf{closure} of $A ,$ denoted $A$ or $C l ( A ) ,$ is the intersection of all closed sets containing $A .$\\
		\\
		The closure of $ A $ is closed and contains $ A $
	\subsection{Theorem: Facts on Previous Definitions}
		Let $X$ be a topological space and $A$ and $B$ be subsets of $X .$\\
		Notice, $ Int(A) \subset A \subset Cl(A) $\\
		\\
		(i) If $U$ is an open set in $X$ and $U \subset A ,$ then $U \subset \ln t ( A )$\\
		(ii) If $C$ is a closed set in $X$ and $A \subset C ,$ then $C l ( A ) \subset C$\\
		(iii) If $A \subset B$ then Int $( A ) \subset Int( B )$\\
		(iv) If $A \subset B$ then $Cl ( A ) \subset Cl ( B )$\\
		(v) $A$ is open if and only if $A = \operatorname { Int } ( A )$\\
		(vi) $A$ is closed if and only if $A = C l ( A )$
	\subsection{Dense}
		A subset $ B $ of a topological space $X$ is called dense if
		$Cl(B) = X .$
	\subsection{Theorem: An element is in the Interior, if the element is contained in an open set}
		Let $X$ be a topological space, $A$ be a subset of $X ,$ and $y$ be an element of $X .$ Then $y \in \operatorname { Int } ( A )$ if and only if there exists an open set $U$ such
		that $y \in U \subset$ A.
	\subsection{Theorem: An element is in the Closure, if the element is contained in the intersection of every open set}
		Let $X$ be a topological space, $A$ be a subset of $X ,$ and $y$ be an element of $X .$ Then $y \in Cl( A )$ if and only if every open set containing $y$ intersects $A .$
	\subsection{Theorem: Facts on Int and Cl with sets}
		For sets $A$ and $B$ in a topological space $X ,$ the following statements hold:\\
		(i) $\ln t ( X - A ) = X - C l ( A )$\\
		(ii) $\operatorname { Cl } ( X - A ) = X - \operatorname { Int } ( A )$\\
		(iii) $\ln t ( A ) \cup \operatorname { Int } ( B ) \subset \operatorname { Int } ( A \cup B ) ,$ and in general equality does not hold.\\
		(iv) $\operatorname { lnt } ( A ) \cap Int ( B ) = \operatorname { Int } ( A \cap B )$
	\subsection{Definition: Limit Point}
		Let $A$ be a subset of a topological space $X .$ A point $x$ in
		$X$ is a \textbf{limit point of} $A$ if every neighborhood of $x$ intersects $A$ in a point other
		than $x .$
	\subsection{Theorem: Limit points provide an easy way to find the closure of a set}
		Let $ A $ be a subset of a topological space $X ,$ and let $A ^ { \prime } b e$
		the set of limit points of $A .$ Then $C l ( A ) = A \cup A ^ { \prime } .$\\
		\subitem Corollary: A subset $  A  $ of a topological space is closed if and only if
		it contains all of its limit points.
	\subsection{Definition: Converge}
		In a topological space $X ,$ a sequence $\left( x _ { 1 } , x _ { 2 } , \ldots \right)$ \textbf{converges to } $x \in X$ if for every neighborhood $U$ of $x ,$ there is a positive integer $N$ such that $x _ { n } \in U$ for all $n \geq N .$ We say that $x$ is the \textbf{limit} of the sequence $\left( x _ { 1 } , x _ { 2 } , \ldots \right) ,$ and we write
		\[\lim _ { n \rightarrow \infty } x _ { n } = x\]
		The idea behind a sequence converging to a point $x$ is that, given any
		neighborhood $U$ of $x ,$ the sequence eventually enters and stays in $U$ .
	\subsection{Theorem: If x is a limit point, there is a sequence that converges to x}
		Let $ A  $ be a subset of $\mathbb { R } ^ { n }$ in the standard topology. If $x$ is a limit point of $A$ , then there is a sequence of points in A that converges to $x .$
	\subsection{Theorem: If $X$ is a Hausdorff space, then every convergent sequence of points in $X$ converges to a unique point in $X .$}
	\subsection{Definition: Boundary}
		Let $A$ be a subset of a topological space $X .$ The \textbf{boundary} of $A ,$ denoted $\partial A ,$ is the set $\partial A = C l ( A ) - \operatorname { In } t ( A )$
	\subsection{Theorem: An element is in the boundary if and only if all neighborhoods of that element are in the subset and the complement of our topology space}
		Let $ A  $be a subset of a topological space $X$ and let $x$ be a point in $X .$ Then $x \in \partial A$ if and only if every neighborhood of $x$ intersects both
		A and$X - A .$
	\subsection{Theorem: Facts on boundaries}
		Let $A$ be a subset of a topological space $X .$ Then the following statements about the boundary of $ A $ hold:\\
		\\
		(i) $\partial A$ is closed.\\
		(ii) $\partial A = C l ( A ) \cap C l ( X - A )$\\
		(iii) $\partial A \cap Int ( A ) = \varnothing$\\
		(iv) $\partial A \cup Int ( A ) = C l ( A )$\\
		(v) $\partial A \subset A$ if and only if $A$ is closed.\\
		(vi) $\partial A \cap A = \varnothing$ if and only if $A$ is open.\\
		(vii) $\partial A = \varnothing$ if and only if $A$ is both open and closed.
		
\section{}

\subsection{Definition: Subspace Topology}
	Let $X$ be a topological space and let $Y$ be a subset of $X .$ Define $T _ { Y } = \{ U \cap Y | U$ is open in $X \} .$ This is called the \textbf{subspace topology} on $Y$ and, with topology, $Y$ is called a \textbf{subspace} of $X .$ We say that $V \subset Y$ is \textbf{open in $Y$} if $V$ is an open set in the subspace topology on $Y .$\\
	\\
	Thus, a set is open in the subspace topology on $Y$ if it is the intersection
	of an open set in $X$ with $Y .$

\subsection{Definition: Standard Topology on $ Y \subset \R^n $}
	Let $ Y $ be a subset of $ \R^n $. The \textbf{standard topology} on $ Y $ is the topology that $Y$ inherits as a subspace of $\mathbb { R } ^ { n }$ with the standard topology.
\subsection{Definition: Closed In $ Y \subset X $}
	Let $X$ be a topological space, and let $Y \subset X$ have the subspace topology. We say that a set $C \subset Y$ is \textbf{closed in $Y$} if $C$ is closed in the subspace topology on $Y .$
\subsection{Theorem: Closed sets are in a subspace $ Y $ if there exists a closed set $ D $ in the superspace} 
	Let $X$ be a topological space, and let $Y \subset X$ have the subspace topology. Then $C \subset Y$ is closed in $Y$ if and only if $C = D \cap Y$ for some closed set $D$ in $X$ .
\subsection{Theorem: Bases for Subspaces}
	Let $X$ be a topological space and $\mathcal { B }$ be a basis for the topology on $X .$ If $Y \subset X ,$ then the collection
		\[\mathcal { B } _ { Y } = \{ B \cap Y | B \in \mathcal { B } \}\]
	is a basis for the subspace topology on $ Y $
\subsection{Definition: Product Topology}
	Let $X$ and $Y$ be topological spaces and $X \times Y$ be their product. The product topology on $X \times Y$ is the topology generated by the basis
	\[\mathcal { B } = \{ U \times V | U \text { is open in } X \text { and } V \text { is open in } Y \}\]
\subsection{Theorem: The collection $\mathcal { B }$ is a basis for a topology on $X \times Y .$}
\subsection{Theorem: The cross product of bases, result in a basis that generates a product topology}
	If $\mathcal { C }$ is a basis for $X$ and $\mathcal { D }$ is a basis for $Y ,$ then
		\[\mathcal { E } = \{ C \times D | C \in \mathcal { C } \text { and } D \in \mathcal { D } \}\]
	is a basis that generates the product topology on $X \times Y .$
\subsection{Theorem: Subspace of a product topology}
	Let $X$ and $Y$ be topological spaces, and assume that $A \subset X$ and $B \subset Y .$ Then the topology on $A \times B$ as a subspace of the product $X \times Y$ is the same as the product topology on $A \times B$ , where $A$ has the subspace topology inherited from $X$ , and $B$ has the subspace topology inherited from Y.
\subsection{Theorem: Cross product of interior is the same as the interior of the cross product}
	Let $A$ and $B$ be subsets of topological spaces $X$ and $Y$ ,
	respectively. Then Int $( A \times B ) = \operatorname { Int } ( A ) \times \operatorname { Int } ( B) .$
\subsection{Definition: Quotient Topology}
	Let $X$ be a topological space and $A$ be a set that is not necessarily a subset of $X$ ). Let $p : X \rightarrow$ A be a surjective map. Define a subset $ U $ of $ A $ to be open in $ A $ if and only if $p ^ { - 1 } ( U )$ is open in $X .$ The resultant collection of open sets in $A$ is called the \textbf{quotient topology induced by} $p$ , and the function $ p $ is called the \textbf{quotient map.} The topological space $ A $ is called a \textbf{quotient space}.
\subsection{Theorem: Let $p : X \rightarrow A$ be a quotient map. The quotient topology on $ A $ is induced by $ p $ is a topology}
\end{document}


