\documentclass[12pt]{article}
\usepackage{color,latexsym,fancyhdr,amsmath,amsfonts,dsfont,amssymb}
\usepackage{color,soul}
\newtheorem{theorem}{Theorem}[section]


\newtheorem{claim}[theorem]{Claim}
\newcommand{\Z}{\mathds{Z}}
\newcommand{\R}{\mathds{R}}
\newcommand{\B}{\mathcal{B}}
\newcommand{\T}{\mathcal{T}}


\topmargin        -0.2 in
\textheight       8.4 in
\oddsidemargin    0 in  
\evensidemargin   0 in     
\textwidth        6.5 in
\headheight       15pt     
\headsep          .35 in     


\begin{document}
\pagestyle{fancy} \lhead{MTH 415 Homework 02} \chead{\textcolor{red}{02/25/2019}}
\rhead{\textcolor{red}{Austin Klum}} 
\lfoot{} \cfoot{} \rfoot{}

\begin{enumerate}
    \item[0.1] Start everyone problem with the homework number followed by the problem written out in its entirety. 
    \item[0.2] After writing down the homework question, begin your solution as a new paragraph. 
    
    \noindent Your solution should be completely self-contained. This means, in part, that everything must be defined in your solution. If you reference a theorem from the textbook, you need to state the theorem name or number.
    
    \item[0.3] Be sure to use the latex math mode when writing mathematical expressions. Mathematicians read x and $x$ differently. 
    
    \item[0.4] Your homework is worth 40\% of your overall course grade. Each assignment will be worth 40 points and there will be 10 assignments. 
    
    \item[0.5] The homework rubric is as follows: 
    \begin{enumerate}
        \item I have the right to return assignments ungraded. This will be done when the the assignment is illegible or the mathematics cannot be followed. You may or may not have an opportunity to redo the assignment.
        \item If I suspect that you are not the author of a solution, you may be asked to explain yourself and justify your response. If you have questions about the originality of your work, please come and see me (More on originality) The ideas in your work should also be your own. While it is allowed to discuss problems with your peers and even use the internet for researching problems, you need to make sure you can reproduce and explain the ideas you use. 
        \item Each problem is worth 5 points.
        \begin{enumerate}
            \item[-] A score of 5/5 means the solution was exemplar
            \item[-] A score of 4/5 means there are minor issues but the solution is overall correct.
            \item[-] A score of 3/5 means that there are errors in your solution but that the general idea behind your proof is sound. 
            \item[-] A score of 2/5 means that there are issues with both your mathematics and logical reasoning. 
            \item[-] A score of 1/5 means that while incorrect, it is clear that effort was put into the solution.
            \item[-] A score of 0/5 means that the problem was not completed or that if the problem was completed it is incorrect and little to no effort was put forth. 
        \end{enumerate}
        \item[0.6] Paying attention to your own learning is critical to mathematical success. Each homework assignment should contain a thoughtful narrative to your problem solving thoughts, ideas, struggles, and successes throughout the completion of this assignment.  This reflection will always be graded and worth 5 points per assignment.
    \end{enumerate}
    \item[0.7] You will need to use this homework template to complete your assignments. 
    \item[0.8] Copy and paste the editing link to the project into Canvas.
\end{enumerate} 
\begin{enumerate}
	\item[1.25]$ X , \text { if } U \text { is open and } C \text { is closed, then } U - C $ is open and $ C-U $ is closed.
	\\ Let $ X $ be a topological space with $ U $ being open and $ C $ being closed.\\
				Notice, $ U-C $ can be rewritten to $ U\cap C^{\complement} $. By definition of closed set we have that the complement is open. Thus, as $ U $ is open and finite intersections of open sets are open we must have that $ U-C $ is open.\\
				Notice, $ C-U $ can be rewritten to $ C\cap U^{\complement} $. By definition of open set we have that the complement is open. Thus, as $ C $ is closed and arbitray unions of closed sets are closed we must have that $ C-U $ is closed.\\
	\item[1.26]Prove that closed balls are closed sets in the standard topology on $ \R^2 $ \\
	If the closed ball $ \B(x,\epsilon)$ is equal to X, then $ \B $ must be closed as the complement is the empty set which by definition must be open. Thus, $ \B $ must be closed. \\
	Suppose that the closed ball $ \B $ is not equal to $ X $ and hence not the empty set.
	Then there must exist an element $ y\in \B^\complement $. Let the $ d(x,y) = h > \epsilon $.\\
	 $[WTS: \B'(y,h-\epsilon) \subset \B^\complement]$.\\
	Suppose for sake of contradiction, the open ball $\B'(y,h-\epsilon) \not\subset \B^\complement$. Then, there exists a $ z\in \B' $ such that $ z\in\B $.\\
	 Notice, $ d(x,y)\leq d(x,z)+d(z,y) $. Which gives us $ d(x,z)\leq \epsilon$, $d(z,y) < h-\epsilon \Rightarrow r+d(z,y)<h-\epsilon+\epsilon<h, $ and $ d(x,z)+d(z,y)<h$.\\
	  Then by transitivity, we have that $d(x,y)<h$. But this is a contradiction as $d(x,y)=h$. Thus, we have that $ z \not\in \B' $ which then gives us $ z\in \B^\complement $. Hence, $ \B' \subset \B^\complement $. And so, $ z $ is an interior point of $ \B^\complement $ and implies that $ \B^\complement $ must be open.\\ Therefore, by definition of closed we must have that $ \B $ is closed.
	\item[1.27]$ \begin{array} { l } { \text { The infinite comb } C \text { is the subset of the plane illustrated in Figure } 1.17 \text { and } } \\ { \text { defined by } } \\ { C = \{ ( x , 0 ) | 0 \leq x \leq 1 \} \cup \left\{ \left( \frac { 1 } { 2 ^ { n } } , y \right) | n = 0,1,2 , \ldots \text { and } 0 \leq y \leq 1 \right\} } \end{array} $\\
		\begin{enumerate}
			\item[(a)] Prove that $C$ is not closed in the standard topology on $\mathbb { R } ^ { 2 }$\\
			 $ [WTS: C^\complement \text{ is not open}] $\\
			Consider the point $ p=(0,1/2)\in C^\complement$ and ball centered at point  $p$ with radius $ \epsilon>0 $. We can find some $ \frac{1}{2^n}<\epsilon$. Hence, the point $ (\frac{1}{2^n},\frac{1}{2}) $ has distance to $ p $ less than $ \epsilon $. Thus, we have $ 0 $ as a limit point of $ C $, but $ (0,\frac{1}{2}) \not\in C$. Hence, $ C $ is not closed in the standard topology on $ \R^2 $.
			\item[(b)] Prove that $C$ is closed in the vertical interval topology on $\mathbb { R } ^ { 2 }$ . \\
			Let $ (x,y)\in U$, where $ U $ is a neighborhood of the form $ \{x\}\times (y-\epsilon,y+\epsilon) $. Assume this is a limit point of $ C $. Then every neighborhood of $ C $ must intersect $ C $. Thus, $ x $ must be of the form $ \frac{1}{2^n} $.Hence, $ C $ contains all it's limit points.\\
			 Therefore, $ C $ is closed in the vertical line topology.
		\end{enumerate}
	\item[1.33] Prove theorem 1.17: Let $ X $ be a topological space.\\
		\begin{enumerate}
			\item[(a)]Prove that $\varnothing$ and $X$ are closed sets.\\
				Notice, $ \phi,X \subseteq X $ and $ \phi^\complement = X $. Similarly, $ X^\complement = \phi $. Thus, $ \phi $ and $ X $ are closed sets.
			\item[(b)]Prove that the intersection of any collection of closed sets in $X$ is a closed set\\
				Let $ \cap_{i\in I} U_i $ be the intersection of a indexed collection of closed sets of $ X $. By DeMorgen's Law, we have $\cap_{i\in I} U_i^c  $. Notice, $ U_i $ is closed for each $ i\in I $ and then we have $ U_i^\complement $ is open for each $ i\in I $. Thus, we have a union of arbitrary open sets which is open. Therefore, $ \cap_{i\in I} U_i $ is closed.
			\item[(c)]Prove that the union of finitely many closed sets in $X$ is a closed set.\\
				Let $ \cup_{i=1}^n U_i $ be a union of a finite number of closed sets in X. By DeMorgen's Law, we have $ \cap_{i=1}^n U_i^\complement $. Since, $ U_i $ is closed, we must have $\cup_i^\complement $ is open. Therefore, the complement is a finite intersection of open sets which is open. Thus, $ \cup_{i=1}^n U_i $ is closed.
		\end{enumerate}
	\item[1.35] Show that $\mathbb { R }$ in the lower limit topology is Hausdorff.\\
		Suppose $ a,b $ are distinct point in the lower limit topology. Assume without loss of generality, $ a<b $. Notice, $ [a,b) $ and $ [b,b+1) $ are disjoint open neighborhoods of $ a $ and $ b $. Therefore, $\R$ is Hausdorff in the lower limit topology.
	\item[1.36] Show that $\mathbb { R }$ in the finite complement topology is not Hausdorff.\\
		By way of contradiction, let $ U,V $ be disjoint open sets. Then, $ V \subset (\R-U) $. Notice, $ \R -U $ is a finite set and so $ V $ is finite. But $ \R-V $ is infinite, which is a contradiction to $ V $ is open. Thus, the opposite must be true and $ U,V $ are not disjoint open sets. Therefore, $ \R $ is not Hausdorff in the finite complement.
	\item[2.02] Prove theorem 2.2: Let $ X $ be a topological space and $ A $ and $ B $ be subsets of $ X $.
		\begin{enumerate}
			\item[(a)]If $C$ is a closed set in $X$ and $A \subset C ,$ then $\mathrm { Cl } ( A ) \subset C$\\
				Let $ C $ be a closed set in $ X $ and $ A\subset C $. It is known that the $ Cl(A) $ is the smallest closed set containing $ A $. Since, $ A \subset C $, we must have that $ Cl(A) \subset C $.
			\item[(b)] If $A \subset B$ then $\mathrm { Cl } ( A ) \subset \mathrm { Cl } ( B )$\\
				Let $ A \subset B $. Notice, that the $ Cl(A) $ and $ Cl(B) $ are the smallest closed sets containing $ A $ and $ B $ respectively. Since, $ A $ is containing in $ Cl(B) $ we must have that $ Cl(A)\subset Cl(B)$
			\item[(c)] $A$ is closed if and only if $A = \mathrm { Cl } ( A )$\\
				Assume $ A $ is closed. Then, $ A $ is in the intersection of all closed sets and $ Cl(A) $ is the smallest closed set containing $ A $. Thus, the intersection will be equal to $ A $. \\
				Therefore $ A=Cl(A) $ \\
				\\
				Suppose $ A=Cl(A) $. Notice, $ Cl(A) $ is closed as the arbitrary intersection of closed sets is closed. \\
				Therefore, $ A $ is closed.
		\end{enumerate}
	\item[2.05] Consider the excluded point topology $E P X _ { p }$ on a set $X$.
	Determine Int $( A )$ and $\mathrm { Cl } ( A )$ for sets $A$ containing $p$ and for sets $A$ not containing $p .$\\
	With $ p $:\\
		$ Cl(A) = p $\\
		$ Int(A) = A-\{p\}$\\
	Without $ p $:\\
		$ Cl(A) = X$\\
		$ Int(A) = A$\\
	\item[2.06] Prove that $\mathrm { Cl } ( \mathbb { Q } ) = \mathbb { R }$ in the standard topology on $\mathbb { R }$ .\\
		Let $ x\in\R $ and $ x\in(a,b) $. In the interval $ (a,b) $, there is a rational. Therefore, $ (a,b)\cap\mathds{Q}=\phi. $. Then, we have $ x\in Cl(\mathds{Q}) \Rightarrow \R = Cl(\mathds{Q})$
	\item[2.07] Let $B = \left\{ \frac { a } { 2 ^ { n } } \in \mathbb { R } | a \in \mathbb { Z } , n \in \mathbb { Z } _ { + } \right\} .$ Show that $B$ is dense in $\mathbb { R }$\\
		Let $ \epsilon > 0 , x\in \R $, and there exist $ a_1,a_2\in B $ such that $ a_1<x<a_2 $. Define $ a_1 = \frac{m-1}{2^n} $ and $ a_2=\frac{m+1}{2^n} $. Notice, $ \frac{m-1}{2^n} < x < \frac{m+1}{2^n} $ implies $ \frac{-1}{2^n} < x - \frac{m}{2^n} < \frac{1}{2^n}$. We can then manipulate this further for $ |x-\frac{m}{2^n} | < \frac{1}{2^n} < \epsilon$. Thus, we have $ Cl(B) = \R $\\
		Therefore, $ B $ is dense in $ \R $
	\item[2.11] 
	Prove Theorem 2.6: For sets $ A $ and $ B $ in a topological space $ X $, the following hold:
	\begin{enumerate}
		\item[(a)] $\mathrm { Cl } ( X - A ) = X - \operatorname { Int } ( A )$\\
			By DeMorgen's Law, $ X- Int(A) = (X - A)^\complement = Cl(X-A).$
		\item[(b)] $\operatorname { Int } ( A ) \cap \operatorname { Int } ( B ) = \operatorname { Int } ( A \cap B )$\\
			 Let $ n\in Int(A) \cap Int(B) $. Then $ n\in Int(A) $ and $ n\in Int(B) $. Which gives us that $ A $ is a neighborhood of $ n $ and $ B $ is a neighborhood of $ n $. Thus, $ A\cap B $ is a neighborhood of $ n $. Hence, $ n \in Int(A\cap B) $. \\
		Therefore, $ Int(A)\cap Int(B) \subseteq Int(A\cap B) $\\
		\\
		Let $ n \in Int(A\cap B) $. Then, $ A\cap B $ is a neighborhood of $ n $. It follows that $ n\in Int(A) $ and $ n\in Int(B) $. Thus, $ n\in Int(A) \cap Int(B) $.\\
		Therefore, $ Int(A\cap B) \subseteq Int(A) \cap Int(B) $,\\
		\\
		Therefore, $ Int(A)\cap Int(B) = Int(A\cap B) $\\
	\end{enumerate}
				 
	\item[2.12] \[Cl(A)\cap Cl(B) \color{red}\supset \color{black} Cl(A\cap B) \]
				\[Cl(A)\cup Cl(B) \color{red}= \color{black} Cl(A\cup B) \]
\end{enumerate}
 
\end{document}


