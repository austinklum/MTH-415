\documentclass[12pt]{article}
\usepackage{color,latexsym,fancyhdr,amsmath,amsfonts,dsfont,amssymb}
\usepackage{color,soul}
\newtheorem{theorem}{Theorem}[section]


\newtheorem{claim}[theorem]{Claim}
\newcommand{\Z}{\mathbb{Z}}
\newcommand{\R}{\mathbb{R}}
\newcommand{\Q}{\mathbb{Q}}
\newcommand{\B}{\mathcal{B}}
\newcommand{\T}{\mathcal{T}}


\topmargin        -0.2 in
\textheight       8.4 in
\oddsidemargin    0 in  
\evensidemargin   0 in     
\textwidth        6.5 in
\headheight       15pt     
\headsep          .35 in     


\begin{document}
	\pagestyle{fancy} \lhead{MTH 415 Homework 08} 
	\chead{04/26/2019}
	\rhead{Austin Klum}
	\lfoot{} \cfoot{} \rfoot{}
	
	\begin{enumerate}
		\item[7.01] Show that every set $A \subset \R$ is a compact subset of $\mathbb { R }$ in the finite complement topology on $\mathbb { R }$ .\\
		Let $ A \subset \R $ and $ \{U_\alpha\} $ be an open cover. Notice, that any set in the cover it's complement has finitely many elements, namely $ x_1,\cdots x_n $	are not in this set. Then, $ \{U_{\alpha_i\}^n_i=1 $ is a finite subcover.\\
		Therefore, $ A $ is a compact subset of $ \R $ in the finite complement topology on $ \R $.	
		\item[7.02] Prove Theorem $7.6 :$ Let $X$ be a topological space.
		\begin{enumerate}
			\item[(a)] If $C _ { 1 } , \ldots , C _ { n }$ are each compact in $X ,$ then $U _ { i = 1 } ^ { n } C _ { j }$ is compact in $X$
			\item[(b)] If $X$ is Hausdorff, and $\left\{ C _ { \alpha } \right\} _ { \alpha \in A }$ is a collection of sets that are compact in
			$X ,$ then $\cap _ { \alpha \in A } C _ { \alpha }$ is compact in $X$ .
		\end{enumerate}
		
		\item[7.03] Provide an example demonstrating that an arbitrary union of compact sets in a
		topological space $X$ is not necessarily compact.
		
		\item[7.07] Recall that the arithmetic progression topology on $Z$ is generated by the basis $\B = \{ A _ { a , b } | a , b \in \Z , b \neq 0 \} ,$ where each
			\[A_{a,b} = \{ \ldots , a - 2 b , a - b , a , a + b , a + 2 b , \ldots \}\]
		is an arithmetic progression. Determine whether or not $ \Z $ is compact in this
		topology.
		\item[7.12] Show that the Tube Lemma does noes not necessarily hold if we drop the assumption that $Y$ is compact. That is, provide an example of a noncompact space $Y$ and
		an open set $U$ in $X \times Y$ such that $U$ contains a slice $\{ x \} \times Y \subset X \times Y$ but does
		not contain an open tube $W \times Y$ containing the slice.
		
		\item[7.17] Use compactness to prove that the plane is not homeomorphic to the sphere.
		(Recall, in Section 6.2 we distinguished between a number of pairs of spaces, including the line and the plane and the line and the sphere, but we indicated that we were not yet in a position to distinguish between the plane and the sphere. With compactness, we can now make that distinction.)
		
		\item[7.18] In this exercise we demonstrate that if we drop the condition that $X$ is Hausdorff in Theorem 7.6 , then the intersection of compact sets in $X$ is not necessarily a compact set. Define the extra-point line as follows. Let $X = \mathbb { R } \cup \left( p _ { e } \right)$, where $p _ { e }$ is an extra point, not contained in $\mathbb { R }$. Let $\B$ be the collection of	subsets of $X$ consisting of all intervals $( a , b ) \subset \mathbb { R }$ and all sets of the form $( c , 0 ) \cup \left\{ p _ { e } \right\} \cup ( 0 , d )$ for $c < 0$ and $d > 0$.\\
		\begin{enumerate}
			\item[(a)] Prove that $\mathcal { B }$ is a basis for a topology on $X .$
			
			\item[(b)] Show that the resulting topology on $X$ is not Hausdorff.
			
			\item[(c)] Find two compact subsets of $X$ whose intersection is not compact. Prove
			that the sets are compact and that the intersection is not.
			
		\end{enumerate}
		
		\item[7.19] \begin{enumerate}
			\item[(a)] Let $( X , d )$ be a metric space. Prove that if $A$ is compact in $X ,$ then $A$ is closed in $X$ and bounded under the metric $d .$
			\item[(b)] Provide an example demonstrating that a subset of a metric space can be
			closed and bounded but not compact.
		\end{enumerate}
	\end{enumerate}
	\section*{Summary}
\end{document}


