\documentclass[12pt]{article}
\usepackage{color,latexsym,fancyhdr,amsmath,amsfonts,dsfont,amssymb}
\usepackage{color,soul}
\newtheorem{theorem}{Theorem}[section]


\newtheorem{claim}[theorem]{Claim}
\newcommand{\Z}{\mathbb{Z}}
\newcommand{\R}{\mathbb{R}}
\newcommand{\Q}{\mathbb{Q}}
\newcommand{\B}{\mathcal{B}}
\newcommand{\T}{\mathcal{T}}


\topmargin        -0.2 in
\textheight       8.4 in
\oddsidemargin    0 in  
\evensidemargin   0 in     
\textwidth        6.5 in
\headheight       15pt     
\headsep          .35 in     


\begin{document}
	\pagestyle{fancy} \lhead{MTH 415 Homework 08} 
	\chead{04/26/2019}
	\rhead{Austin Klum}
	\lfoot{} \cfoot{} \rfoot{}
	
	\begin{enumerate}
		\item[7.01] Show that every set $A \subset \R$ is a compact subset of $\mathbb { R }$ in the finite complement topology on $\mathbb { R }$ .\\
		Let $ A \subset \R $ and $ \{U_\alpha\} $ be an open cover. Notice, that any set in the cover it's complement has finitely many elements, namely $ x_1,\cdots x_n $	are not in this set. Then, $ \{U_\{\alpha_i\}^n_i=1 $ is a finite subcover.\\
		Therefore, $ A $ is a compact subset of $ \R $ in the finite complement topology on $ \R $.	
		\item[7.02] Prove Theorem $7.6 :$ Let $X$ be a topological space.
		\begin{enumerate}
			\item[(a)] If $C _ { 1 } , \ldots , C _ { n }$ are each compact in $X ,$ then $U _ { j = 1 } ^ { n } C _ { j }$ is compact in $X$\\
			Let $ \{C_1,\ldots,C_n\} $ be a collection of compact subspaces of $ X $. We define $ C = \cup^n_{j=1} C_j$. Suppose $ O $ is a cover for $ C $. Then, notice each $ C_j $ is compact and so has a finite subcover $ O_j $. We then will have $ O' = \cup^n_{j=1} O_j $. \\
		    Thus, $ O' $ is an open cover for $ C $.\\
			Therefore, $ C $ is compact.
			\item[(b)] If $X$ is Hausdorff, and $\left\{ C _ { \alpha } \right\} _ { \alpha \in A }$ is a collection of sets that are compact in
			$X ,$ then $\cap _ { \alpha \in A } C _ { \alpha }$ is compact in $X$ .\\
			Notice, that each $ C_j $ in the collection is closed since it's in a Hausdorff space. Thus, the finite intersection of the collection is also closed. Since every $ C_\alpha $ lives inside $ A $ for some $ \alpha\in A $, we also have that the collection is bounded. Since the collection is both closed and bounded, we must have the collection is compact.
		\end{enumerate}
		
		\item[7.03] Provide an example demonstrating that an arbitrary union of compact sets in a
		topological space $X$ is not necessarily compact.\\
		Let $ X $ be an infinite set with the discrete topology. Notice, the collection of singletons gives an open cover with no subcover. Thus, an arbitrary union of compact sets in a topological space $X$ is not necessarily compact.
		
		\item[7.12] Show that the Tube Lemma does noes not necessarily hold if we drop the assumption that $Y$ is compact. That is, provide an example of a noncompact space $Y$ and
		an open set $U$ in $X \times Y$ such that $U$ contains a slice $\{ x \} \times Y \subset X \times Y$ but does not contain an open tube $W \times Y$ containing the slice.\\
		Consider the space $ \R $ and open set defined as $ U = \{(x,y)||x\cdot y < 1\} $. Notice, $ U $ contains $ \{0\}\times \R $, but cannot contain the a tube. But for $ U $ to contain $ \{0\} $, we would have $ U =\{0\} $ which is not open.\\
		 Thus, we must have $ Y $ to compact for the Tube Lemma to hold true.
		
		\item[7.17] Use compactness to prove that the plane is not homeomorphic to the sphere.
		(Recall, in Section 6.2 we distinguished between a number of pairs of spaces, including the line and the plane and the line and the sphere, but we indicated that we were not yet in a position to distinguish between the plane and the sphere. With compactness, we can now make that distinction.)\\
		Notice the sphere is compact. Suppose that there exists a continuous bijection from the sphere to the plane. Thus, the plane would have to be compact since we've supposed there exists a continuous function mapping the sphere to the plane. This is a contradiction as the plane is not compact. Since, the sphere is compact and the plane is not, we cannot have a homeomorphism.
		
		\item[7.18] In this exercise we demonstrate that if we drop the condition that $X$ is Hausdorff in Theorem 7.6 , then the intersection of compact sets in $X$ is not necessarily a compact set. Define the extra-point line as follows. Let $X = \mathbb { R } \cup \left( p _ { e } \right)$, where $p _ { e }$ is an extra point, not contained in $\mathbb { R }$. Let $\B$ be the collection of	subsets of $X$ consisting of all intervals $( a , b ) \subset \mathbb { R }$ and all sets of the form $( c , 0 ) \cup \left\{ p _ { e } \right\} \cup ( 0 , d )$ for $c < 0$ and $d > 0$.\\
		\begin{enumerate}
			\item[(a)] Prove that $\mathcal { B }$ is a basis for a topology on $X .$
			
			\item[(b)] Show that the resulting topology on $X$ is not Hausdorff.
			
			\item[(c)] Find two compact subsets of $X$ whose intersection is not compact. Prove
			that the sets are compact and that the intersection is not.
			
		\end{enumerate}
		
		\item[7.19] \begin{enumerate}
			\item[(a)] Let $( X , d )$ be a metric space. Prove that if $A$ is compact in $X ,$ then $A$ is closed in $X$ and bounded under the metric $d .$\\
			Suppose $ A $ is compact in $ X $.  Consider $\{B(0,n)|n\in \mathbb{N}\} $. Notice this is an open cover for $ X $. This must also be an open cover for $ A $ since $ A $ is a subset of $ X $. Thus, $ A $ is bounded.\\
			Let $ x \in A^\complement $. For every $ y \in A $, there are open neighborhoods of  $y, U_y $ and $ V_y $ of $ x $ such that $ U_y \cap V_y \varnothing $. Then, $ \{U_y|y\in A\} $ is an open cover of $ A. $. Since, $ A $ is compact we then have that $ U $ and $ Y $ are open and $ U\cap V = \varnothing$. We then have that $ A^\complement$ is open. Thus, $ A $ is closed.\\
			Therefore, if $A$ is compact in $X ,$ then $A$ is closed and bounded.
			\item[(b)] Provide an example demonstrating that a subset of a metric space can be
			closed and bounded but not compact.\\
			
			Let $ X $ be the integers and let our metric be defined as such:
				\[d(x,y) = \{\begin{array} {ll} {1} & {\text{ if  $x \neq y$ }} \\ {0} & {\text { if $x = y$}} \end{array}\]
			Notice, $ d(x,y) $ is bounded since each point is within a distance 1 of some other point. Notice, every subset of $ X $ is open and thus also closed. Thus, we are bounded and closed. It is not compact as there are no finite subcovers, since $ X $ is infinite.
		\end{enumerate}
	\end{enumerate}
	\section*{Summary}
\end{document}


