\documentclass[12pt]{article}
\usepackage{color,latexsym,fancyhdr,amsmath,amsfonts,dsfont,amssymb}
\usepackage{color,soul}
\newtheorem{theorem}{Theorem}[section]


\newtheorem{claim}[theorem]{Claim}
\newcommand{\Z}{\mathbb{Z}}
\newcommand{\R}{\mathbb{R}}
\newcommand{\B}{\mathcal{B}}
\newcommand{\T}{\mathcal{T}}


\topmargin        -0.2 in
\textheight       8.4 in
\oddsidemargin    0 in  
\evensidemargin   0 in     
\textwidth        6.5 in
\headheight       15pt     
\headsep          .35 in     


\begin{document}
\pagestyle{fancy} \lhead{MTH 415 Homework 06} 
\chead{04/05/2019}
\rhead{Austin Klum}
\lfoot{} \cfoot{} \rfoot{}

\begin{enumerate}

	\item[5.01] Show that the taxicab metric on $\mathbb { R } ^ { 2 }$ satisfies the properties of a metric.
	\begin{enumerate}
		\item[(1)] Notice, by the definition of the Taxicab metric we take the addition of two absolute values. Since absolute values are never negative, we must have that for some $ x,y\in \R^2 , d(x,y)\geq 0$. Note, if $ x=y $, we must have that $ d(x,y)=0 $ and if $ x\not=y, d(x,y)>0 $\\
		Thus, property 1 is satisfied.
		\item[(2)] Let $ x,y\in \R^2 $. Observe.
		\begin{align*}
			d(x,y) &= |x_1-y_1|+|x_2-y_2|\\
				   &= |y_1-x_1|+|y_2-x_2|\\
				   &= d(y,x)
		\end{align*}
		Thus, property 2 is satisfied.
		\item[(3)] Let $ x,y,z \in \R^2 $. Observe.
			\begin{align*}
				d(x,z) &= |x_1-z_1|+|x_2-z_2|\\
					   &= |x_1-y_1+y_1-z_1|+|x_2-y_2+y_2-z_2|\\
					   &\leq |x_1-y_1|+|y_1-z_1|+|x_2-y_2|+|y_2-z_2|\\
					   &=    |x_1-y_1|+|x_2-y_2|+|y_1-z_1|+|y_2-z_2|\\
					   &= d(x,y)+d(y,z)
			\end{align*}
		Thus, property 3 is satisfied.
	\end{enumerate}
	Therefore, the taxicab metric is a metric.
	\item[5.02] 
	\begin{enumerate}
		\item[(a)] Show that the max metric on $\mathbb { R } ^ { 2 }$ satisfies the properties of a metric.
		\begin{enumerate}
			\item[(1)] Notice, we are taking the max value of an absolute value. Since absolute values are never negative, we must have that for some $ x,y\in \R^2 , d(x,y)\geq 0$. Note, if $ x=y $, we must have that $ d(x,y)=0 $ and if $ x\not=y, d(x,y)>0 $\\ 
			Thus, property 1 is satisfied.
			\item[(2)] Let $ x,y\in \R^2 $. Observe.
			\begin{align*}
			d(x,y) &= max\{|x_1-y_1|,|x_2-y_2|\}\\
			&= max\{|y_1-x_1|,|y_2-x_2|\}\\
			&= d(y,x)
			\end{align*}
			Thus, property 2 is satisfied.
			\item[(3)] Let $ x,y,z \in \R^2 $. Observe.
			\begin{align*}
			d(x,z)  &= max\{|x_1-z_1|,|x_2-z_2|\}\\
					&= max\{|x_1-y_1+y_1-z_1|,|x_2-y_2+y_2-z_2|\}\\
					&\leq \max\{|x_1-y_1|+|y_1-z_1|,|x_2-y_2|+|y_2-z_2|\}\\
					&= |x_i-y_i|+|y_i-z_i| \\
					&\text{ where $ i $ with value 1 or 2 holds the maximum value}\\
			|x_i-y_i|&\leq max\{|x_1-y_1|,|x_2-y_2|\}\\
			|y_i-z_i|&\leq max\{|y_1-z_1|,|y_2-z_2|\}
			\end{align*}
			So, 
				\[d(x,z) \leq max\{|x_1-y_1|,|x_2-y_2|\} + max\{|y_1-z_1|,|y_2-z_2|\}=d(x,y)+d(y,z)\]
			Thus, property 3 is satisfied.
		\end{enumerate}	
		Therefore, the max metric is a metric.
%		\item[(b)] Explain why $d ( p , q ) = \min \left\{ \left| p _ { 1 } - q _ { 1 } \right| , \left| p _ { 2 } - q _ { 2 } \right| \right\}$ does not define a metric on $\mathbb { R } ^ { 2 }$ .\\
		The Triangle inequality does not hold.\\
		Let $ p,q,r \in \R^2 $. Observe.
		\begin{align*}
			d(p,r) &= min\{|p_1-r_1|,|p_2-r_2|\}\\
				   &\geq min\{|p_1-q_1|+|q_1-r_1|,|p_2-q_2|+|q_2-r_2|\}\\
				   &= |p_i-q_i|+|q_i-r_i|\\
				   &\text{ where $ i $ with value 1 or 2 holds the minimum value}\\
				    |p_i-q_i| &\leq min\{|p_1-q_1|,|p_2-q_2|\}\\
				    |q_i-r_i| &\leq min\{|q_1-r_1|,|q_2-r_2|\}
		\end{align*}
		So, 
			\[d(p,r) \leq min\{|p_1-q_1|,|p_2-q_2|\} + min\{|q_1-r_1|,|q_2-r_2|\}\leq d(p,q)+d(q,r)\]
	\end{enumerate}
	\item[5.03] For points $p = \left( p _ { 1 } , p _ { 2 } \right)$ and $q = \left( q _ { 1 } , q _ { 2 } \right)$ in $\mathbb { R } ^ { 2 }$ define
		\[d _ { V } ( p , q ) = \left\{ \begin{array} { l l } { 1 } & { \text { if } p _ { 1 } \neq q _ { 1 } \text { or } \left| p _ { 2 } - q _ { 2 } \right| \geq 1 } \\ { \left| p _ { 2 } - q _ { 2 } \right| } & { \text { if } p _ { 1 } = q _ { 1 } \text { and } \left| p _ { 2 } - q _ { 2 } \right| < 1 } \end{array} \right.\]
	\begin{enumerate}
		\item[(a)] Show that $d _ { V }$ is a metric.
	
		\item[(b)] 	Describe the open balls in the metric $d _ { V }$ .
	\end{enumerate}
	\item[5.10]
	\begin{enumerate}
		\item[(a)] Let $( X , d )$ be a metric on a space. For $x , y \in X ,$ define
		\[D ( x , y ) = \frac { d ( x , y ) } { 1 + d ( x , y ) }\]
		Show that $D$ is also a metric on $ X $
		
		\item[(b)] Explain why no two points in $X$ are distance one or more apart in the
		metric $D .$
	\end{enumerate}
	
	\item[5.24] Prove Theorem $5.13 :$ Let $\left( X , d _ { X } \right)$ and $\left( Y , d _ { Y } \right)$ be metric spaces. A function
	$f : X \rightarrow Y$ is continuous in the open set definition if and only if for each $x \in X$ and $\varepsilon > 0 ,$ there exists a $\delta > 0$ such that if $x ^ { \prime } \in X$ and $d x \left( x , x ^ { \prime } \right) < \delta$
	then $d _ { Y } \left( f ( x ) , f \left( x ^ { \prime } \right) \right) < \varepsilon .$ (Hint: Consider Exercise 4.3 and the proof of
	Theorem $4.6 .$ )
	
	\item[5.25] Let $( X , d )$ be a metric space, and assume $p \in X$ and $A \subset X$
	\begin{enumerate}
		\item[(a)] Provide an example showing that $d ( \{ p \} , A ) = 0$ need not imply that $p \in A .$
		
		\item[(b)] Prove that if $A$ is closed and $d ( \{ p \} , A ) = 0 ,$ then $p \in A$
		
	\end{enumerate}
	
	\item[5.26] Use Theorem 5.15 to prove that the taxicab metric and the max metric induce
	the same topology on $\mathbb { R } ^ { 2 }$ .
	
	\item[5.28] Let $( X , d )$ be a metric space. The function
		\[D ( x , y ) = \frac { d ( x , y ) } { 1 + d ( x , y ) }\]
	is a bounded metric on $X$ . (See Exercise 5.10.) Show that the topologies
	induced by $D$ and $d$ are the same.
	
	\item[5.29] On the set of continuous functions $C [ a , b ]$ consider the metrics $\rho _ { M }$ and $\rho$ defined by
	\[\rho _ { M } ( f , g ) = \max _ { x \in [ a , b ] } [ | f ( x ) - g ( x ) \| \} ,\] and
	\[\rho ( f , g ) = \int _ { a } ^ { b } | f ( x ) - g ( x ) | d x\]
	These metrics were introduced in Exercise 5.8 and Example 5.5, respectively.
	\begin{enumerate}
		\item[(a)] Use Theorem 5.15 to prove that the topology induced by $\rho _ { M }$ on $C [ a , b ]$ is finer than the topology induced by $\rho$ .
		\item[(b)] Show that for every $c _ { 1 } , c _ { 2 } > 0$ there exists $f \in C [ a , b ]$ such that\\
		$\max _ { x \in [ a , b ] } \{ | f ( x ) | \} = c _ { 1 }$ and
		\[\int _ { a } ^ { b } | f ( x ) | d x = c _ { 2 }\]
		\item[(c)] Let $Z \in C [ a , b ]$ be the function defined by $Z ( x ) = 0$ for all $x \in [ a , b ]$
		Given $\varepsilon > 0 ,$ show that no $\delta > 0$ exists such that $B _ { \rho } ( Z , \delta ) \subset B _ { \rho M} ( Z , \varepsilon )$
		(Hint: Part (b) helps.)
		\item[(d)] What does Theorem 5.15 allow us to conclude from (c)?
	\end{enumerate}

	\section*{Summary}

\end{enumerate}
 
\end{document}


