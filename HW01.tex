\documentclass[12pt]{article}
\usepackage{color,latexsym,fancyhdr,amsmath,amsfonts,dsfont,amssymb}
\usepackage{color,soul}
\newtheorem{theorem}{Theorem}[section]


\newtheorem{claim}[theorem]{Claim}



\topmargin        -0.2 in
\textheight       8.4 in
\oddsidemargin    0 in  
\evensidemargin   0 in     
\textwidth        6.5 in
\headheight       15pt     
\headsep          .35 in     


\begin{document}
\pagestyle{fancy} \lhead{MTH 415 Homework} \chead{\textcolor{red}{02/15/2019}}
\rhead{\textcolor{red}{Austin Klum}} 
\lfoot{} \cfoot{} \rfoot{}

\begin{enumerate}
    \item[0.1] Start everyone problem with the homework number followed by the problem written out in its entirety. 
    \item[0.2] After writing down the homework question, begin your solution as a new paragraph. 
    
    \noindent Your solution should be completely self-contained. This means, in part, that everything must be defined in your solution. If you reference a theorem from the textbook, you need to state the theorem name or number.
    
    \item[0.3] Be sure to use the latex math mode when writing mathematical expressions. Mathematicians read x and $x$ differently. 
    
    \item[0.4] Your homework is worth 40\% of your overall course grade. Each assignment will be worth 40 points and there will be 10 assignments. 
    
    \item[0.5] The homework rubric is as follows: 
    \begin{enumerate}
        \item I have the right to return assignments ungraded. This will be done when the the assignment is illegible or the mathematics cannot be followed. You may or may not have an opportunity to redo the assignment.
        \item If I suspect that you are not the author of a solution, you may be asked to explain yourself and justify your response. If you have questions about the originality of your work, please come and see me (More on originality) The ideas in your work should also be your own. While it is allowed to discuss problems with your peers and even use the internet for researching problems, you need to make sure you can reproduce and explain the ideas you use. 
        \item Each problem is worth 5 points.
        \begin{enumerate}
            \item[-] A score of 5/5 means the solution was exemplar
            \item[-] A score of 4/5 means there are minor issues but the solution is overall correct.
            \item[-] A score of 3/5 means that there are errors in your solution but that the general idea behind your proof is sound. 
            \item[-] A score of 2/5 means that there are issues with both your mathematics and logical reasoning. 
            \item[-] A score of 1/5 means that while incorrect, it is clear that effort was put into the solution.
            \item[-] A score of 0/5 means that the problem was not completed or that if the problem was completed it is incorrect and little to no effort was put forth. 
        \end{enumerate}
        \item[0.6] Paying attention to your own learning is critical to mathematical success. Each homework assignment should contain a thoughtful narrative to your problem solving thoughts, ideas, struggles, and successes throughout the completion of this assignment.  This reflection will always be graded and worth 5 points per assignment.
    \end{enumerate}
    \item[0.7] You will need to use this homework template to complete your assignments. 
    \item[0.8] Copy and paste the editing link to the project into Canvas.
\end{enumerate} 
\begin{enumerate}
	\item[1.1] Determine all of the possible topologies on  $ X =\{a,b\}$
	
	\item[1.2]$  \begin{array} { l } { \text { On the three-point set } X = \{ a , b , c \} , \text { the trivial topology has two open sets and } } \\ { \text { the discrete topology has eight open sets. For each of } n = 3 , \ldots , 7 , \text { either find } } \\ { \text { a topology on } X \text { consisting of } n \text { open sets or prove that no such topology exists. } } \end{array} $
	
	\item[1.3]$  \begin{array} { l } { \text { Prove that a topology } \mathcal { T } \text { on } X \text { is the discrete topology if and only if } \{ x \} \in \mathcal { T } } \\ { \text { for all } x \in X . } \end{array} $
	
	\item[1.4] \begin{enumerate}
				\item $ \begin{array} { l } { \text { Give an example of a space where the discrete topology is the same as the } } \\ { \text { finite complement topology. } } \end{array} $
				\item$  \begin{array} { l } { \text { Make and prove a conjecture indicating for what class of sets the discrete } } \\ { \text { and finite complement topologies coincide. } } \end{array}$
			\end{enumerate}
	
	\item[1.5] $ \begin{array} { l } { \text { Find three topologies on the five-point set } X = \{ a , b , c , d , e \} \text { such that the the } } \\ { \text { first is strictly finer than the seecond and the second strictly finer than the third, } } \\ { \text { without using either the trivial or the discrete topology. Find a topology on } X } \\ { \text { that is not comparable to each of the first three that you found. } } \end{array} $
	
	\item[1.8]$  \begin{array} { l } { \text { Let } X \text { be a set and assume } p \in X . \text { Show that the collection } \mathcal { T } , \text { consisting of } X } \\ { \text { and all subsets of } X \text { that exclude } p , \text { is a topology on } X . \text { This topology is called } } \\ { \text { the excluded point topology on } X , \text { and we denote it by } E P X _ { p } \text { . } } \end{array} $
	
	\item[1.9]$  \begin{array} { l } { \text { Let } \mathcal { T } \text { consist of } \varnothing, \mathbb { R } , \text { and all intervals } ( - \infty , p ) \text { for } p \in \mathbb { R } . \text { Prove that } \mathcal { T } \text { is a } } \\ { \text { topology on } \mathbb { R } \text { . } } \end{array} $
	
	\item[1.10] Show that $ \mathcal { B } = \{ [ a , b ) \subset \mathbb { R } | a < b \} \text { is a basis for a topology on } \mathbb { R } $
	
	\item[1.11]$ \begin{array} { l } { \text { Determine which of the following collections of subsets of } \mathbb { R } \text { are bases: } } \\ { \text { (a) } \mathcal { C } _ { 1 } = \{ ( n , n + 2 ) \subset \mathbb { R } | n \in \mathbb { Z } \} } \\ { \text { (b) } \mathcal { C } _ { 2 } = \{ [ a , b ] \subset \mathbb { R } | a < b \} } \\ { \text { (c) } \mathcal { C } _ { 3 } = \{ [ a , b ] \subset \mathbb { R } | a \leq b \} } \\ { \text { (d) } \mathcal { C } _ { 4 } = \{ ( - x , x ) \subset \mathbb { R } | x \in \mathbb { R } \} } \\ { \text { (e) } \mathcal { C } _ { 5 } = \{ ( a , b ) \cup \{ b + 1 \} \subset \mathbb { R } | a < b \} } \end{array} $
	
	\item[1.13]$ \begin{array} { l } { \text { Consider the following six topologies defined on } \mathbb { R } \text { : the trivial topology, the } } \\ { \text { discrete topology, the finite complement topology, the standard topology, the } } \\ { \text { lower limit topology, and the upper limit topology. Show how they compare } } \\ { \text { to each other (finer, strictly finer, coarser, strictly coarser, noncomparable) and } } \\ { \text { justify your claim. } } \end{array} $
	
	\item[1.16]$ \begin{array} { c } { \text { Prove Theorem } 1.12 : \text { On the plane } \mathbb { R } ^ { 2 } , \text { let } } \\ { \mathcal { B } = \left\{ ( a , b ) \times ( c , d ) \subset \mathbb { R } ^ { 2 } | a < b , c < d \right\} } \end{array} \begin{array} { l } { \text { (a) Show that } \mathcal { B } \text { is a basis for a topology on } \mathbb { R } ^ { 2 } \text { . } } \\ { \text { (b) Show that the topology, } \mathcal { T } ^ { \prime } \text { , generated by } \mathcal { B } \text { is the standard topology on } } \end{array} \mathbb { R } ^ { 2 } \text { (Hint: if } \mathcal { T } \text { is the standard topology, show that } \mathcal { T } \subset \mathcal { T } ^ { \prime } \text { and } \mathcal { T } ^ { \prime } \subset \mathcal { T } . ) $
	
	\item[1.17] $ \begin{array} { l } { \text { An open half plane is a subset of } \mathbb { R } ^ { 2 } \text { in the form } \left\{ ( x , y ) \in \mathbb { R } ^ { 2 } | A x + B y < C \right\} } \\ { \text { for some } A , B , C \in \mathbb { R } \text { with either } A \text { or } B \text { nonzero. (See Figure } 1.11 . \text { ) Prove } } \\ { \text { that open half planes are open sets in the standard topology on} \mathbb { R }^2 \text { . } } \end{array} $
	
	\item[1.18]$ \begin{array} { l } { \text { Show that the collection } \{ ( - \infty , q ) \subset \mathbb { R } | q \text { rational } \} \text { is a basis for the topology } } \\ { \text { in Exercise } 1.9 . } \end{array} $
	
	\item[1.19]$ \begin{array} { l } { \text { (a) Show that the collection } \left\{ \{ a \} \times ( b , c ) \subset \mathbb { R } ^ { 2 } | a , b , c \in \mathbb { R } \right\} \text { of vertical } } \\ { \text { intervals in the plane is a basis for a topology on } \mathbb { R } ^ { 2 } . \text { We call this topology } } \\ { \text { the vertical interval topology. } } \end{array} \text {(b) Compare the vertical interval topology with the standard topology on} \mathbb {R}^2 $
\end{enumerate}
 
\end{document}


