\documentclass[12pt]{article}
\usepackage{color,latexsym,fancyhdr,amsmath,amsfonts,dsfont,amssymb}
\usepackage{color,soul}
\newtheorem{theorem}{Theorem}[section]


\newtheorem{claim}[theorem]{Claim}
\newcommand{\Z}{\mathds{Z}}
\newcommand{\R}{\mathds{R}}
\newcommand{\B}{\mathcal{B}}
\newcommand{\T}{\mathcal{T}}


\topmargin        -0.2 in
\textheight       8.4 in
\oddsidemargin    0 in  
\evensidemargin   0 in     
\textwidth        6.5 in
\headheight       15pt     
\headsep          .35 in     


\begin{document}
\pagestyle{fancy} \lhead{MTH 415 Homework} \chead{\textcolor{red}{02/15/2019}}
\rhead{\textcolor{red}{Austin Klum}} 
\lfoot{} \cfoot{} \rfoot{}

\begin{enumerate}
    \item[0.1] Start everyone problem with the homework number followed by the problem written out in its entirety. 
    \item[0.2] After writing down the homework question, begin your solution as a new paragraph. 
    
    \noindent Your solution should be completely self-contained. This means, in part, that everything must be defined in your solution. If you reference a theorem from the textbook, you need to state the theorem name or number.
    
    \item[0.3] Be sure to use the latex math mode when writing mathematical expressions. Mathematicians read x and $x$ differently. 
    
    \item[0.4] Your homework is worth 40\% of your overall course grade. Each assignment will be worth 40 points and there will be 10 assignments. 
    
    \item[0.5] The homework rubric is as follows: 
    \begin{enumerate}
        \item I have the right to return assignments ungraded. This will be done when the the assignment is illegible or the mathematics cannot be followed. You may or may not have an opportunity to redo the assignment.
        \item If I suspect that you are not the author of a solution, you may be asked to explain yourself and justify your response. If you have questions about the originality of your work, please come and see me (More on originality) The ideas in your work should also be your own. While it is allowed to discuss problems with your peers and even use the internet for researching problems, you need to make sure you can reproduce and explain the ideas you use. 
        \item Each problem is worth 5 points.
        \begin{enumerate}
            \item[-] A score of 5/5 means the solution was exemplar
            \item[-] A score of 4/5 means there are minor issues but the solution is overall correct.
            \item[-] A score of 3/5 means that there are errors in your solution but that the general idea behind your proof is sound. 
            \item[-] A score of 2/5 means that there are issues with both your mathematics and logical reasoning. 
            \item[-] A score of 1/5 means that while incorrect, it is clear that effort was put into the solution.
            \item[-] A score of 0/5 means that the problem was not completed or that if the problem was completed it is incorrect and little to no effort was put forth. 
        \end{enumerate}
        \item[0.6] Paying attention to your own learning is critical to mathematical success. Each homework assignment should contain a thoughtful narrative to your problem solving thoughts, ideas, struggles, and successes throughout the completion of this assignment.  This reflection will always be graded and worth 5 points per assignment.
    \end{enumerate}
    \item[0.7] You will need to use this homework template to complete your assignments. 
    \item[0.8] Copy and paste the editing link to the project into Canvas.
\end{enumerate} 
\begin{enumerate}
	\item[1.1] Determine all of the possible topologies on  $ X =\{a,b\}$\\
			$ \mathcal{ T } = \{\phi, X\},\{\phi,X,\{a\},\{b\}\},\{\phi,X,\{a\}\},\{\phi,X,\{b\}\}   $
	\item[1.2]$  \begin{array} { l } { \text { On the three-point set } X = \{ a , b , c \} , \text { the trivial topology has two open sets and } } \\ { \text { the discrete topology has eight open sets. For each of } n = 3 , \ldots , 6 , \text { either find } } \\ { \text { a topology on } X \text { consisting of } n \text { open sets or prove that no such topology exists. } } \end{array} $\\
	$n=3 \quad \mathcal{ T }= \{\phi,X,\{a\}\} $\\
	$n=4 \quad \mathcal{ T }= \{\phi,X,\{a\},\{b\}\} $\\
	$n=5 \quad \mathcal{ T }= \{\phi,X,\{a\},\{b\},\{a,b\}\} $\\
	$n=6 \quad \mathcal{ T }= \{\phi,X,\{a\},\{b\},\{a,b\},\{b,c\}\} $\\
	
	\item[1.3]$  \begin{array} { l } { \text { Prove that a topology } \mathcal { T } \text { on } X \text { is the discrete topology if and only if } \{ x \} \in \mathcal { T } } \\ { \text { for all } x \in X . } \end{array} $\\
	Suppose $ (X,\T) $ is a topological space and let $ \{x\}\in \T ,\forall x\in X$. Notice, by the definition of a topological space we have that $ \phi,X \in \T$. \\
	Let $ A\subseteq X $. Then $ \forall y\in A \subset X $ we have $ \{y_i\}\in \T $. Observe.
		\[\cup_{i\in\Z}\{y_i\}\in\T\Rightarrow A\in\T\]
	Thus, arbitrary unions of elements in $ \T $ are in $ \T $.\\
	Notice.
		\[\cap_{i\in\Z}\{y_i\}\in\T\Rightarrow A\in\T\]
	Thus, finite unions of elements in $\T $ are in $ \T $.\\
	Therefore, we have that $ \T $ on $ X $ is the discrete topology.
	\\
	Suppose we have the discrete topology $ \T $. Then by the definition of the discrete topology, we have that $ x\in\T \forall \{x\}\in \T $.
	\item[1.4] \begin{enumerate}
				\item $ \begin{array} { l } { \text { Give an example of a space where the discrete topology is the same as the } } \\ { \text { finite complement topology. } } \end{array} $
				An example would be a set which is finite. e.g. $ X=\{1,2,3\} $
				\item$  \begin{array} { l } { \text { Make and prove a conjecture indicating for what class of sets the discrete } } \\ { \text { and finite complement topologies coincide. } } \end{array}$\\
				$ \underline{Claim:} $ The sets that coincide are the finite sets.\\
				Suppose that $ X $ is a finite set and $ Y \subseteq X $. Then we have $ X-Y $ is finite and thus $ Y $ is open in the finite complement topology. \\
				Thus, the discrete topologies that coincide are finite sets.
			\end{enumerate}
	
	\item[1.5] $ \begin{array} { l } { \text { Find three topologies on the five-point set } X = \{ a , b , c , d , e \} \text { such that the the } } \\ { \text { first is strictly finer than the seecond and the second strictly finer than the third, } } \\ { \text { without using either the trivial or the discrete topology. Find a topology on } X } \\ { \text { that is not comparable to each of the first three that you found. } } \end{array} $\\
	$ \mathcal{ T }_1 = \{\phi,X,\{a\}\} $\\
	$ \mathcal{ T }_2 = \{\phi,X,\{a\}\{a,b\}\} $\\
	$ \mathcal{ T }_3 = \{\phi,X,\{a\}\{a,b\},\{a,b,c\}\} $\\
	$ \mathcal{ T }_4 = \{\phi,X,\{a\}\{a,b\},\{a,b,c\},\{b\}\} $\\
	\item[1.8]$  \begin{array} { l } { \text { Let } X \text { be a set and assume } p \in X . \text { Show that the collection } \mathcal { T } , \text { consisting of } X } \\ { \text { and all subsets of } X \text { that exclude } p , \text { is a topology on } X . \text { This topology is called } } \\ { \text { the excluded point topology on } X , \text { and we denote it by } E P X _ { p } \text { . } } \end{array} $\\
	Let $ X $ be a set and assume $ p\in X $. $ \mathcal{ T } $ consists of $ X $ and all subsets of $ X $ that exclude $ p $.
	\begin{enumerate}
		\item[i] By the definition of $ \mathcal{ T } $, we have $ \phi,X\in\mathcal{ T } $
		\item[ii] As the intersection of all the sets that exclude $ p $ will all have sets in common, we then have that the finite intersections of the open sets that exclude $ p $ are open.
		\item[iii] As the union of the open sets excluding $ p $ will be in $ \mathcal{ T } $ and since we will never union on $ p $, then the unions must be open sets.
	\end{enumerate}
	
	\item[1.9]$ \text { Let } \mathcal { T } \text { consist of } \varnothing, \mathbb { R } , \text { and all intervals } ( - \infty , p ) \text { for } p \in \mathbb { R } . \text { Prove that } \mathcal { T } \text { is a } \text { topology on } \mathbb { R } \text { . } $\\
	Let $ p \in \mathbb{ R } $ and $ \mathcal{ T } $ consists of $ \phi,\mathbb{R}, $ and all intervals $ (-\infty,p) $. 
	\begin{enumerate}
		\item[i] By definition, $ \phi,\mathbb{ R } \in \mathcal{ T } $
		\item[ii] As we are bounded by the upper value $ p $, we must have that all intersection of intervals will be in the range of $ (-\infty,p) $
		\item[iii] As we are bounded by the upper value $ p $, we must have that all unions of intervals will be in the range of $ (-\infty,p) $
	\end{enumerate}
	
	\item[1.10] Show that $ \mathcal { B } = \{ [ a , b ) \subset \mathbb{ R } | a < b \} \text { is a basis for a topology on } \mathbb { R } $\\
		Let $ x\in \mathbb{ R } $ then $ x\in [x-\epsilon,x+\epsilon),\epsilon > 0 $.\\
		Suppose $ x\in [a_1,b_1)\cap[a_2,b_2) $, then let $ a = $ max $ \{a_1,a_2\} $ and $ b = $ min $ \{b_1,b_2\} $\\
		Thus, $ \mathcal{ B } $ is a basis.
	
	\item[1.11]$ \begin{array} { l } { \text { Determine which of the following collections of subsets of } \mathbb { R } \text { are bases: } } \\ { \text { (a) } \mathcal { C } _ { 1 } = \{ ( n , n + 2 ) \subset \mathbb { R } | n \in \mathbb { Z } \} } \\ { \text { (b) } \mathcal { C } _ { 2 } = \{ [ a , b ] \subset \mathbb { R } | a < b \} } \\ { \text { (c) } \mathcal { C } _ { 3 } = \{ [ a , b ] \subset \mathbb { R } | a \leq b \} } \\ { \text { (d) } \mathcal { C } _ { 4 } = \{ ( - x , x ) \subset \mathbb { R } | x \in \mathbb { R } \} } \\ { \text { (e) } \mathcal { C } _ { 5 } = \{ ( a , b ) \cup \{ b + 1 \} \subset \mathbb { R } | a < b \} } \end{array} $\\
	$ \mathcal{C}_2,\mathcal{C}_3,\mathcal{C}_4,\mathcal{C}_5 $ are all bases.
	\item[1.13]Consider the following six topologies defined on $ \mathbb {R}$: the trivial topology, the discrete topology, the finite complement topology, the standard topology, the lower limit topology, and the upper limit topology. Show how they compare to each other (finer, strictly finer, coarser, strictly coarser, noncomparable) and justify your claim.
	Starting from fine to coarse we have:\\
	Discrete Topology;Finite Complement Topology;Lower Limit Topology (equal in fineness/coarseness) Upper Limit Topology;Standard Topology;Trivial Topology
	\item[1.16] $\text { Prove Theorem } 1.12 : \text { On the plane } \mathbb { R } ^ { 2 } , \text { let } $
		\[\mathcal { B } = \left\{ ( a , b ) \times ( c , d ) \subset \mathbb { R } ^ { 2 } | a < b , c < d \right\}\]
	
	\begin{enumerate}
		\item[(a)] Show that $ \mathcal { B } \text { is a basis for a topology on } \mathbb { R } ^ { 2 } \text { . } $\\
		\begin{enumerate}
			\item[i] Let $ (x,y)\in R^2 $ and $ B \in \mathcal{ B } $ where $ \epsilon_1, \epsilon_2, \epsilon_3, \epsilon_4 > 0 $ and $ B = \{(x-\epsilon_1,x+\epsilon_2)\times(y-\epsilon_3,y+\epsilon_4)\} $. Thus, $ (x,y) \in B $
			\item[ii] Let $ B_1, B_2 \in \mathcal{ B }$ and $ x\in B_1\cap B_2 $. Suppose $ B_1 = \{(a_1,b_1)\times(c_1,d_1)\} $  and $ B_2 = \{(a_2,b_2)\times(c_2,d_2)\}  $ Then let $ B_3 = \{(max(a_1,a_2),max{b_1,b_2})\times(min(c_1,c_2),min(d_1,d_2))\} $. Thus, $ x\in B_3 $.
		\end{enumerate}
		Therefore, $ \mathcal{ B } $ is a basis.
		\item[(b)]  Show that the topology,  $ \mathcal { T } ^ { \prime } \text { , generated by } \mathcal { B } \text { is the standard topology on } \mathbb { R } ^ { 2 } \text { (Hint: if } \mathcal { T } \text { is the } \\ \text{standard topology, show that } \mathcal { T } \subset \mathcal { T } ^ { \prime } \text { and } \mathcal { T } ^ { \prime } \subset \mathcal { T }$. \\
		Let $ x \in \mathcal{ B }_1 = \{B(x,\epsilon_1)|x\in\mathbb{ R },\epsilon_1>0\} $. Suppose $ \mathcal{B}_2 = \{(x-\epsilon_1-\epsilon_2,x+\epsilon_1+\epsilon_2)\times (x-\epsilon_1-\epsilon_2,x+\epsilon_1+\epsilon_2)\} $. Notice, that $ x \in \mathcal{B_2} $ by definition, that is the open ball of radius $ \epsilon_1 $ centered at $ x $ is contained by $ \mathcal{ B }_2 $ as we have defined $ \mathcal{ B }_2 $ to be larger.\\
		
		Let $ x \in \mathcal{B}_1 = \{(x-\epsilon_1,x+\epsilon_1)\times (x-\epsilon_1,x+\epsilon_1)\} $. Suppose $  \mathcal{ B }_2 = \{B(x,\epsilon_1+\epsilon_2)|x\in\mathbb{ R },\epsilon_1,\epsilon_2>0\}$. Notice, that $ x \in \mathcal{B}_2 $ by definition, that is the open ball of radius $ \epsilon_1 + \epsilon_2 $ centered at $ x $ contains $ \mathcal{ B }_1 $ as we have defined $ \mathcal{ B }_2 $ to be larger.\\
		\\
		Therefore, as the two are subsets of each either they must be, in fact, equal.
	\end{enumerate}
	
	\item[1.17] $ \begin{array} { l } { \text { An open half plane is a subset of } \mathbb { R } ^ { 2 } \text { in the form } \left\{ ( x , y ) \in \mathbb { R } ^ { 2 } | A x + B y < C \right\} } \\ { \text { for some } A , B , C \in \mathbb { R } \text { with either } A \text { or } B \text { nonzero. (See Figure } 1.11 . \text { ) Prove } } \\ { \text { that open half planes are open sets in the standard topology on} \mathbb { R }^2 \text { . } } \end{array} $\\
	Let $ H = \{(x,y)\in\R^2|Ax+By<C\} $ where $ x>0 $. Now, we need to construct an epsilon ball such that we contain $ H $. There are three cases for such,
	\begin{enumerate}
		\item[Case 1:] $ A \not= 0, B\not= 0 $\\
			Then we have that $ H = \{(x,y)\in\R^2|\frac{x}{A}+\frac{y}{B}<\frac{c}{AB}\} $. Notice, we can form an epsilon ball around any point $ (a,b)\in H $ by $ B_\epsilon (a,b) $ where $ \epsilon > 0 $  
		\item[Case 2:] $ A = 0, B \not= 0 $
			Then we have that $ H = \{(x,y)\in\R^2|y <\frac{C}{B}\} $. Notice, we can form an epsilon ball around any point $ (a,b)\in H $ by $ B_\epsilon (a,b) $ where $ \epsilon > 0 $  
		\item[Case 3:] $ A \not= 0, B = 0 $
			Then we have that $ H = \{(x,y)\in\R^2|x<\frac{C}{A}\} $. Notice, we can form an epsilon ball around any point $ (a,b)\in H $ by $ B_\epsilon (a,b) $ where $ \epsilon > 0 $  
	\end{enumerate}
 	Therefore, for all cases of $ A $ and $ B $ we are able to construct an epsilon ball that contains $ H $. Thus, the open half planes are open sets in the standard topology in $ \R^2 $
	\item[1.18]$ \begin{array} { l } { \text { Show that the collection } \{ ( - \infty , q ) \subset \mathbb { R } | q \text { rational } \} \text { is a basis for the topology } } \\ { \text { in Exercise } 1.9 . } \end{array} $\\
	Let $ x\in\R $ and suppose $ B\in \B $ where $ B = \{(-\inf,q)|q \text( rational and) x<q\}$. Thus, $ x\in B $.\\
	Let $ B_1 = \{(\inf,q_1)|q_1 \text( rational)\} $, $ B_2= \{(\inf,q_2)|q_2 \text( rational)\} $ and suppose $ x\in B_1\cap B_2 $. Notice, either $ q_1\leq q_2 $ or $ q_2 \leq q_1 $. Without loss of generality suppose $ q_1 \leq q_2 $. Then $ B_1\cap B_2 $ = $ B_1  \Rightarrow x\in B_1 \subset B_1\cap B_2$. Thus, there exists a $ B_3 $ where $ x\in B_3 $, namely $ B_3 = B_1 $.\\
	Therefore, $ \B $ satisfies both conditions for being a basis for a topology.
	
	\item[1.19] 
	\begin{enumerate}
		\item[(a)]  Show that the collection $ \left\{ \{ a \} \times ( b , c ) \subset \mathbb { R } ^ { 2 } | a , b , c \in \mathbb { R } \right\} \text { of vertical } \text { intervals in the plane is a} \\ \text{ basis for a topology on } \mathbb { R } ^ { 2 } . \text { We call this topology } \text { the vertical interval topology. } $\\
		Let $ x\in \R $. Then let $ B\in \mathcal {B} $ where $ B = \{a\} \times (x-\epsilon_1,x+\epsilon_2) $ for $ \epsilon_1, \epsilon_2 > 0 $. Thus, $ x \in B. $\\
		Suppose $ x  \in \{a\}\times(b_1,c_1)\cap\{a\}\times(b_2,c_2)\} $. Then let $ B_3 = \{a\} \times $ (max$ \{b_1,c_1\} $, min$ \{b_2,c_2\} $)
		\item[(b)] Compare the vertical interval topology with the standard topology on $ \mathbb {R}^2 $\\
		Notice, as the standard topology is a subset of $ \R^2 $, we also have that the vertical interval topology is a subset of $ \R^2 $. Based off of the definition of both, we then have that the vertical interval topology is a subset of the standard topology.
	\end{enumerate}
\end{enumerate}
 
\end{document}


