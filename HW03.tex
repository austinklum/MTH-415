\documentclass[12pt]{article}
\usepackage{color,latexsym,fancyhdr,amsmath,amsfonts,dsfont,amssymb}
\usepackage{color,soul}
\newtheorem{theorem}{Theorem}[section]


\newtheorem{claim}[theorem]{Claim}
\newcommand{\Z}{\mathds{Z}}
\newcommand{\R}{\mathds{R}}
\newcommand{\B}{\mathcal{B}}
\newcommand{\T}{\mathcal{T}}


\topmargin        -0.2 in
\textheight       8.4 in
\oddsidemargin    0 in  
\evensidemargin   0 in     
\textwidth        6.5 in
\headheight       15pt     
\headsep          .35 in     


\begin{document}
\pagestyle{fancy} \lhead{MTH 415 Homework 03} 
\chead{03/01/2019}
\rhead{Austin Klum}
\lfoot{} \cfoot{} \rfoot{}

\begin{enumerate}
	\item[2.14] For each $n \in \mathbb { Z } _ { + } ,$ let $B _ { n } = \{ n , n + 1 , n + 2 , \ldots \} ,$ and consider the collection $\mathcal { B } = \left\{ B _ { n } | n \in \mathbb { Z } _ { + } \right\}$
	\begin{enumerate}
		\item[(a)] Show that $\mathcal { B }$ is a basis for a topology on $\mathbb { Z } _ { + }$\\
			Let $ x\in \Z_+ $. Notice, $ x \in B_x := \{x,x+1,x+2,\cdots\} $\\
			Thus, every point in $ \Z_+ $ is contained in a basis element.
			Let $ a,b\in\Z , B_a := \{a,a+1,a+2,\cdots\} $ and $ B_b := \{b,b+1,b+2,\cdots\} $ with $ y\in B_a \cup B_b $. Suppose $ m=max(a,b) $. Then, $ y\in B_m\subset B_a \cup B_b$\\
			Thus, every point in the intersection of two basis elements is contained in a basis element contained in that intersection.\\
			Therefore, $ \B $ is a basis on $ \Z_+ $
		\item[(b)] Show that the topology on $X$ generated by $\mathcal { B }$ is not Hausdorff.\\
		Let $ X $ be a set with $ \B $ as a basis for $ X $ and let $ x,y\in X $. Without loss of generality, assume $ x<y $ with basis elements of the form $ B_x := \{x,\cdots,y,y+1,y+2,\cdots\} $ and $ B_y := \{y,y+1,y+2,\cdots\} $ Notice, $ B_x \cap B_y = B_y$.\\
		Thus, the basis are not disjoint.\\
		Therefore, the topology generated by $ \B $ is not Hausdorff.
		\item[(c)] Show that the sequence $( 2,4,6,8 , \dots )$ converges to every point in $\mathbb { Z } _ { + }$ with the topology generated by $\mathcal { B }$
		Let $ j\in \Z_+ $. Suppose $ U $ is a neighborhood of $ j $. Suppose $ k = 2j $. Then for all elements of $ (2,4,6,c\dots) \geq 2j $, are in $ U $. \\ Therefore, the sequence $ (2,4,6,\cdots) $ converges to every point in $ \Z_+ $ with the topology generate by $ \B $.
		\item[(d)] Prove that every injective sequence converges to every point in $\mathbb { Z } _ { + }$ with the topology generated by $\mathcal { B }$\\
		Let $ s $ be an injective sequence and $ z\in\Z_+ $. Notice, that $ s = B_z $. Thus, the every injective sequence converges to every point in $ \Z_+ $.\\
		\textcolor{red}{[To be honest, I have no idea what this is asking of me.]}
	\end{enumerate}
	\item[2.15] Determine the set of limit points of $[ 0,1 ]$ in the finite complement topology on $ \R $\\
	Notice, $ [0,1] $ is an infinite subset of $ \R $. Let $x\in\R $ and $U$ be a neighborhood of $ x $. Then $ [0,1]\cap U \not = \varnothing$ and is infinite. Thus, the limit points of $ [0,1] $ is every point.
	\item[2.17] 
		\begin{enumerate}
			\item[(a)] Let $\mathcal { B } = \{ [ a , b ) \subset \mathbb { R } | a , b \in \mathbb { Q }$ and $a < b \} .$ Show that $\mathcal { B }$ is a basis for a topology on $\mathbb { R }$ . The resulting topology is called the rational lower limit topology and is denoted $\mathbb { R } _ { r l }$\\
			Let $ x\in\R $. Suppose $ B \in \B $ such that $ x\in B := [x-\epsilon,x+\epsilon) $ for some $ \epsilon > 0 $.\\
			Thus, every point in $ \Z_+ $ is contained in a basis element.\\
			\\
			Let $ B_1=[a,b) $ and $ B_2 = [c,d) $ such that $ x \in B_1\cap B_2 $. Let $ x = max(a,c) $ and $ y = min(b,d) $. Notice, $ x \in B = [x,y) \subset B_1\cap B_2 $. \\
			Thus, every point in the intersection of two basis elements is contained in a basis element contained in that intersection.\\
			\\
			Therefore, $ \B $ is a basis
			\item[(b)] Determine the closures of $A = ( 0 , \sqrt { 2 } )$ and $B = ( \sqrt { 2 } , 3 )$ in $\mathbb { R } _ { l }$ and in $\mathbb { R } _ { r l }$\\
			\underline{Lower Limit:}
			\begin{align*}
				 Cl(A) &= [0,\sqrt{2})\\
				 Cl(B) &= [\sqrt{2},3)
			\end{align*}
			\underline{Rational Lower Limit:}
			\begin{align*}
			Cl(A) &= [0,\sqrt{2}]\\
			Cl(B) &= [\sqrt{2},3)
			\end{align*}
		\end{enumerate}
	\item[2.21] Determine the set of limit points of the set
		\[S = \left\{ \left( x , \sin \left( \frac { 1 } { x } \right) \right) \in \mathbb { R } ^ { 2 } | 0 < x \leq 1 \right\}\]
		as a subset of $\mathbb { R } ^ { 2 }$ in the standard topology. (The closure of $S$ in the plane is known as the topologist's sine curve.\\
		Let $ y\in[-1.1] $ and $ p  = (0,y)$. Notice, for every neighborhood $ U - \{p\}$ contain points in $ S $ . Thus, every point in $ S $ is a limit point.
	\item[2.27] Determine $\partial ( [ 0,1 ] )$ in $\mathbb { R }$ with the finite complement topology. Justify your result.\\
	Let $ A = [0,1] $. We then have $ Cl(A) = \R $ and $ Int(A)=\varnothing $. Hence, $ \partial A = Cl(A)-Int(A)=\R $.\\
	Therefore,  $\partial ( [ 0,1 ] )$ in $\mathbb { R }$ with the finite complement topology is $ \R $
	\item[2.28] Prove Theorem $2.15 :$ Let $A$ be a subset of a topological space $X .$
		\begin{enumerate}
			\item[(a)] $\partial A$ is closed.\\
			Observe, \begin{align*}
		\partial A &= Cl(A) - Int(A) \\
				   &= Cl(A)\cap(X-Int(A))\\
			\end{align*}
			Notice, $ Cl(A) $ is closed and the complement of $ Int(A) $ is closed. \\
			Thus, as intersections of closed sets are closed, we have $ \partial A $ is closed.
			\item[(b)] $\partial A = \mathrm { Cl } ( A ) \cap \mathrm { Cl } ( X - A )$
			Observe,
			 \begin{align*}
			 \partial A &= Cl(A)- Int(A)\\
			 			&= Cl(A) \cap( X - Int(A)) \\
			 			&= Cl(A)\cap Cl(X-A)
			\end{align*}
			Thus,  $\partial A = \mathrm { Cl } ( A ) \cap \mathrm { Cl } ( X - A )$
			\item[(c)] $\partial A \cap \operatorname { In } t ( A ) = \varnothing$\\
			As $ \partial A = Cl(A)-Int(A) $, we have already removed all elements of $ Int(A) $.\\
			Therefore,  $\partial A \cap \operatorname { In } t ( A ) = \varnothing$
			\item[(d)] $\partial A \cup \operatorname { Int } ( A ) = \mathrm { Cl } ( A )$\\
			Notice, 
			\begin{align*}
				\partial A \cup Int(A) &= (Cl(A) - Int(A)) U Int(A) \\
									   &= Cl(A)
			\end{align*}
			Therefore,  $\partial A \cup \operatorname { Int } ( A ) = \mathrm { Cl } ( A )$
			\item[(e)] $\partial A \subset A$ if and only if $A$ is closed.\\
			Let $ \partial A \subset A $. Then, $ A $ must be closed as $ \partial A = Cl(A) - Int(A) $\\
			\\
			Let $ A $ be closed. Then, we have that $ Cl(A) $ is closed. Thus, $ \partial A = Cl(A)- Int(A) $.\\
			Hence, $ \partial A \subset A $.\\
			\\
			Therefore,  $\partial A \subset A$ if and only if $A$ is closed.\\
			\item[(f)] $\partial A \cap A = \varnothing$ if and only if $A$ is open.\\
			Let $ \partial A \cap A = \varnothing $. By way of contradiction, assume $ A $ is not open. Then, there exists a $ x \in A $ such that no open set containing $ x $ is a subset of $ A $. This is a contradiction as $ Int(A) $ is open and $ Int(A)\subset A $.\\
			Thus, $ A $ must be open.\\
			Let $ A $ be open. Then, 
			\begin{align*}
			 \partial A \cap A &= (Cl(A) - Int(A)) \cap A\\
			 				   &= (Cl(A)\cap A^\complement) \cap A \\
			 				   &= Cl(A) \cap (A^\complement \cap A) \\
			 				   &= Cl(A) \cap \varnothing \\
			 				   &= \varnothing
			\end{align*}
			\item[(g)] $\partial A = \varnothing$ if and only if $A$ is both open and closed.\\
			Let $ \partial A = \varnothing $. Notice, $ Int(A) \subset A \subset Cl(A) $. From this we have $ Int(A) = A = Cl(A) $. Which shows that $ A $ is both opened and closed by each part of the equality, respectively.\\
			\\
			Let $ A $ be opened and closed. Then, we have $ A = Int(A) $ and $ A = Cl(A) $. Notice, $ Cl(A) = Int(A) \cup \partial A \Rightarrow Cl(A) = A \cup \partial A $\\
			And then, $ A = A \cup \partial A  $ and $ Int(A) \cap \partial A = \varnothing $. So, $ Int(A) = \partial A = \varnothing. $ \\
			Thus, $ A = A \cup \partial A $ and $ A \cap \partial A = \varnothing$.\\
			Therefore, $ \partial A = \varnothing $
		\end{enumerate}
\end{enumerate}
 
\end{document}


