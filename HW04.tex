\documentclass[12pt]{article}
\usepackage{color,latexsym,fancyhdr,amsmath,amsfonts,dsfont,amssymb}
\usepackage{color,soul}
\newtheorem{theorem}{Theorem}[section]


\newtheorem{claim}[theorem]{Claim}
\newcommand{\Z}{\mathds{Z}}
\newcommand{\R}{\mathds{R}}
\newcommand{\B}{\mathcal{B}}
\newcommand{\T}{\mathcal{T}}


\topmargin        -0.2 in
\textheight       8.4 in
\oddsidemargin    0 in  
\evensidemargin   0 in     
\textwidth        6.5 in
\headheight       15pt     
\headsep          .35 in     


\begin{document}
\pagestyle{fancy} \lhead{MTH 415 Homework 04} 
\chead{03/08/2019}
\rhead{Austin Klum}
\lfoot{} \cfoot{} \rfoot{}

\begin{enumerate}

	\item[3.01] Let $X = \left\{ ( x , 0 ) \in \mathbb { R } ^ { 2 } | x \in \mathbb { R } \right\} ,$ the $x$ -axis in the plane. Describe the topology that $X$ inherits as a subspace of $\mathbb { R } ^ { 2 }$ with the standard topology.\\
	Notice, the basis for the Product Topology is the product of the basis of the individual properties. For $ \R^2 $, let $ a,b,c,d\in\R $, then we have elements of the form $ (a,b)\times(c,d) $. When intersected with $ X $, we result in the subspace topology.

	\item[3.02] Let $Y = [ - 1,1 ]$ have the standard topology. Which of the following sets are open in $Y$ and which are open in $\mathbb { R } ?$\\
		\[A = ( - 1 , - 1 / 2 ) \cup ( 1 / 2,1 ) \text{ ::  Open in $ Y $ and $ \R $}\]
		\[B = ( - 1 , - 1 / 2 ] \cup [ 1 / 2,1 ) \text{ ::  Not open in $ Y $ and $ \R $}\]
		\[C = [ - 1 , - 1 / 2 ) \cup ( 1 / 2,1 ] \text{ ::  Open in $ Y $ and not open in $ \R $}\]
		\[D = [ - 1 , - 1 / 2 ] \cup [ 1 / 2,1 ] \text{ ::  Closed in $ Y $ and $ \R $}\]
		\[E = \bigcup _ { n = 1 } ^ { \infty } \left( \frac { 1 } { 1 + n } , \frac { 1 } { n } \right) \text{ ::  Open in $ Y $ and $ \R $}\]

	\item[3.03] \textbf{Prove Theorem $3.4 :$ }Let $X$ be a topological space, and let $Y \subset X$ have the subspace topology. Then $C \subset Y$ is closed in $Y$ if and only if $C = D \cap Y$ for some closed set $D$ in $X$.\\
	Let $ C\subset Y $ be closed in $ Y $. Then, there exists a closed subset $ D $ in $ X $ with $ D\cap Y = C $. Define $ U = X-D $ and $ V=Y-C $. Note, $ U $ is open. Observe,
		\begin{align*}	
			U\cap Y &= (X-D)\cap Y\\
					&= (X\cap Y)-(D\cap Y)\\
					&= Y-C\\
					&= V
		\end{align*}
	Thus, $ V $ is open. Hence, $ C=Y-V $ is closed.\\
	Therefore, there exists a closed set in $ Y $ such that $ C $ equals the intersection of such a set and $ Y $.
	\\
	Let $ D $ be a closed set in $ X $ and $ C= D\cap Y $. Then, this implies that $ C=D-Y^\complement $. Notice, as $ D $ is closed and $ Y^\complement $ is closed, we must have that $ C $ is closed by definition of the intersection of closed sets.

	\item[3.07] Let $X$ be a Hausdorff topological space, and $Y$ be a subset of $X .$ Prove that the subspace topology on $Y$ is Hausdorff.\\
	Let $ a,b\in Y $ with $ a \not= b $. Since $ X $ is Hausdorff, there exist disjoint neighborhoods $ U $ and $ V $ in $ X $ of $ a $ and $ b $, respectively. Notice, a set containing $ a $ in $ Y $ is $ W_1=U\cap Y $. Then, $ W_1 $ must be open in $ Y $ by the definition of the subspace topology. Hence, $ W_1 $ is a neighborhood of $ a $ in $ Y $. \\
	Notice, a set containing $ b $ in $ Y $ is $ W_2=V\cap Y $. Then, $ W_2 $ must be open in $ Y $ by definition of the subspace topology. Hence, $ W_2 $ is a neighborhood of $ b $ in $ Y $.\\
	Observe that as $ U $ and $V$ are disjoint and$ W_1 \subset U $ and $ W_2 \subset V $, we must have that $ W_1 $ and $ W_2 $ are also disjoint.\\
	Therefore, $ Y $ is Hausdorff.

	\item[3.08] Let $X$ be a topological space, and let $Y \subset X$ have the subspace topology.
		\begin{enumerate}
			\item[(a)] If $A$ is open in $Y ,$ and $Y$ is open in $X ,$ show that $A$ is open in $X .$\\
			Let $ A $ be open in $ Y $ and $ Y $ be open in $ X $. Then, $ A=Y\cap U $ for some open set $ U $ in $ X $. Since, $ Y $ and $ U $ are both open in $ X $, their intersection must also be open by definition.\\
			Therefore, $ A $ is open in $ X $.
			
			\item[(b)] If $A$ is closed in $Y ,$ and $Y$ is closed in $X ,$ show that $A$ is closed in $X .$\\
				Let $ A $ be closed in $ Y $ and $ Y $ be closed in $ X $. Then, $ A=Y\cap U $ for some closed set $ U $ in $ X $. Since, $ Y $ and $ U $ are both closed in $ X $, their intersection must also be closed by definition.\\
				Therefore, $ A $ is open in $ X $.
		\end{enumerate}

	\item[3.15] \textbf{Prove Theorem $3.9 :$} Let $X$ and $Y$ be topological spaces, and assume that $A \subset X$ and $B \subset Y .$ Then the topology on $A \times B$ as a subspace of the product $X \times Y$ is the same as the product topology on $A \times B ,$ where $A$ has the subspace topology inherited from $X ,$ and $B$ has the subspace topology inherited from $Y .$

	\item[3.16] Let $S ^ { 2 }$ be the sphere, $D$ be the disk, $T$ be the torus, $S ^ { 1 }$ be the circle, and $I = [ 0,1 ]$ with the standard topology. Draw pictures of the product spaces
	$S ^ { 2 } \times I , T \times I , S ^ { 1 } \times I \times I ,$ and $S ^ { 1 } \times D$

	\item[3.18] Show that if $X$ and $Y$ are Hausdorff spaces, then so is the product space $X \times Y$

	\item[3.19] Show that if $A$ is closed in $X$ and $B$ is closed in $Y ,$ then $A \times B$ is closed in $X \times Y .$

	\item[3.20] Show that if $A \subset X$ and $B \subset Y ,$ then $\operatorname { Cl } ( A \times B ) = C l ( A ) \times C l ( B )$

	\item[3.23] If $ \R $ has the standard topology, define \\
	$p : \mathbb { R } \rightarrow \{ a , b , c , d , e \}$ by $p ( x ) = \left\{ \begin{array} { l } { a \text { if } x > 2 } \\ { b \text { if } x > 2 } \\ { b \text { if } x = 2 } \\ { d \text { if } 0 \leq x < 2 } \\ { d \text { if } - 1 < x < 0 } \\ { e \text { if } x \leq - 1 } \end{array} \right.$
	\begin{enumerate}
		\item[(a)] List the open sets in the quotient topology on $\{ a , b , c , d , e \}$
		
		\item[(b)] Now assume that $\mathbb { R }$ has the lower limit topology. What are the open sets in the resulting quotient topology on $\{ a , b , c , d , e \} ?$
		
	\end{enumerate}

	\item[3.24] Let $X = \mathbb { R }$ in the standard topology. Take the partition
		\[X ^ { * } = \{ \ldots , ( - 1,0 ] , ( 0,1 ] , ( 1,2 ] , \ldots \}\]
	Describe the open sets in the resulting quotient topology on $X ^ { * }$ .

	\item[3.25] Define a partition of $X = \mathbb { R } ^ { 2 } - \{ O \}$ by taking each ray emanating from the origin as an element in the partition. (See Figure $3.25 .$ Which topological space that we have previously encountered appears to be topologically equivalent to the quotient space that results from this partition?

	\item[3.27] Provide an example showing that a quotient space of a Hausdorff space need
	not be a Hausdorff space.

	\item[3.29] Consider the equivalence relation on $\mathbb { R } ^ { 2 }$ defined by $\left( x _ { 1 } , x _ { 2 } \right) \sim \left( w _ { 1 } , w _ { 2 } \right)$ if $x _ { 1 } + x _ { 2 } = w _ { 1 } + w _ { 2 } .$ Describe the quotient space that results from the partition of $\mathbb { R } ^ { 2 }$ into the equivalence classes in this equivalence relation.

	\item[3.30] Consider the equivalence relation on $\mathbb { R } ^ { 2 }$ defined by $\left( x _ { 1 } , x _ { 2 } \right) \sim \left( w _ { 1 } , w _ { 2 } \right)$ if $x _ { 1 } ^ { 2 } + x _ { 2 } ^ { 2 } = w _ { 1 } ^ { 2 } + w _ { 2 } ^ { 2 } .$ Describe the quotient space that results from the partition of $\mathbb { R } ^ { 2 }$ into the equivalence classes in this equivalence relation.

	\item[3.35] On a sketch of the surface $T \# T$ , illustrate where the glued edges of the octagon in Figure 3.33 appear.

	\item[3.36] \begin{enumerate}
		\item[(a)] Show that a hexagon with opposite edges glued together straight across yields a torus.
			
		\item[(b)]Show that a hexagon with opposite edges glued together with a flip yields	a projective plane.
		
	\end{enumerate}
	
	\item[3.38] Show that the quotient space in Example 3.27 is topologically equivalent to
	$S ^ { 1 } \times P ,$ the product of a circle and a projective plane.

\end{enumerate}
 
\end{document}


