\documentclass[12pt]{article}
\usepackage{color,latexsym,fancyhdr,amsmath,amsfonts,dsfont,amssymb,ragged2e}
\usepackage{color,soul}
\newtheorem{theorem}{Theorem}[section]


\newtheorem{claim}[theorem]{Claim}
\newcommand{\Z}{\mathds{Z}}
\newcommand{\R}{\mathds{R}}
\newcommand{\B}{\mathcal{B}}
\newcommand{\T}{\mathcal{T}}


\topmargin        -0.2 in
\textheight       8.4 in
\oddsidemargin    0 in  
\evensidemargin   0 in     
\textwidth        6.5 in
\headheight       15pt     
\headsep          .35 in     

\begin{document}
\pagestyle{fancy} \lhead{History of Game Theory} \chead{}
\rhead{\text{Austin Klum}} 
\lfoot{} \cfoot{} \rfoot{}

\section{Game Theory: What is it?}
 
\begin{quotation}
	 Game Theory can be defined as the study of mathematical models of
	conflict and cooperation between intelligent rational decision-makers.
	Game Theory provides general mathematical techniques for analyzing
	situations in which two or more individuals make decisions that will
	influence one another's welfare. (Myerson, R.B. 1)
\end{quotation}
	Game Theory is the mathematics behind rational decision making based on the game being "played". This study of games, which are merely formal models of interactive situations, can be "played" in numerous applications such as economics, biology, artificial intelligence, politics, and many other topics. These games are then modeled and the best possible outcome is determined from all options and variations given. Involved in games are players that are often assumed to be rational. That is, it is assumed all players will have all information needed and are able to make their decision based on facts and figures as opposed to emotions and relationships. Mathematics is a beautiful way of describing the world and by using Game Theory we can describe the beauty of decision making. First hinted at in ancient times with deciding fairness and best course of action, Game Theory has grown beyond it's initial implication of games and can now be applied to a majority of decisions.
\section{Brief History}
	Game Theory has had a long informal history. There has always been discussion on the best possible course of action for a set of rules, weather the scenario is a life-or-death situation or more light hearted such as played a board game. The first formalization of Game Theory occurred in early 20th century with mathematician Jon Von Neumann and his paper "On the Theory of Parlor Games" published in 1928. In this paper he began to develop the ideas that will eventually become Game Theory. These ideas were further elaborated in 1944 in the book co-authored by Jon Von Neumann and economist Oskar Morgenstern called "Theory of Games and Economic Behavior". This work truly pushed Game Theory into an actual area of study with formal rules and ideologies. Game Theory was originally discussed as zero-sum games, where one's gain causes the loss for another. Eventually, in further papers and exploration of Game Theory, zero-sum games are a special case rather than the norm. 
	\subsection{Nash Equilibrium}
	 John Nash was a famous mathematician who further advanced Game Theory greatly. John Nash was a able to use these ideas presented by John Von Nuemann in 1950 to show that finite games always have a Nash Equilibrium. 
	 \begin{quotation}
		In Game Theory, the Nash equilibrium is a solution concept of a non-cooperative game involving two or more players in which each player is assumed to know the equilibrium strategies of the other players, and no player has anything to gain by changing only their own strategy (Cai, Liu, and Qu)
 	\end{quotation}
 	Proof of Nash Equilibrium:\\
 	\\
 	Proof: The proof of Nash makes use of Kakutani's fixed point theorem, which states: Let $X$ be a convex closed bounded body in $\Re ^ { n } . $ Then, for all $ x \in X ,$
 	$f ( x )$ is a nonempty convex subset of $X . \{ \langle x , f ( x ) \rangle \}$ is closed. Then $\exists x ^ { * }$ such that $x ^ { * } \in f \left( x ^ { * } \right) .$. Suppose there are 3
 	players, $A , B ,$ and $C .$ Let $\overline { p } , \overline { q } ,$ and $\overline { r }$ be their probability distributions over the action sets. And $\alpha ,$
 	$\beta ,$ and $\gamma$ are their payoff functions. $P ( \overline { q } , \overline { r } )$ denotes the set of best-play $\vec { p } ^ { \prime } s .$ Not hard to see $P ( \overline { q } , \overline { r } )$
 	is a convex closed set. Similarly, we define
 	 $Q ( \overline { r } , \overline { p } )$ and
 	  $R ( \overline { p } , \overline { q } ) .$ 
 	  Define function\[f : \left( \begin{array} { c } { \overline { p } } \\ { \overline { q } } \\ { \overline { r } } \end{array} \right) \rightarrow \left( \begin{array} { c } { P ( \overline { q } , \overline { r } ) } \\ { Q ( \overline { r } , \overline { p } ) } \\ { R ( \overline { p } , \overline { q } ) } \end{array} \right)\]
 	\\
 	then if we could apply Kakutani's fix point theorem, we have\\
 	\[\left( \begin{array} { c } { \overline { p } } \\ { \overline { q } } \\ { \overline { r } } \end{array} \right) \in f \left( \begin{array} { c } { \overline { p } } \\ { \overline { q } } \\ { \overline { r } } \end{array} \right)\]
 	which shows the existence of Nash equilibrium. Now what we need to do is just verify our setup
 	satisfies the conditions of Kakutani's fix point theorem. The first two conditions are trivially
 	satisfied. We only need to show $\{ \langle x , f ( x ) \rangle \}$ is closed, which means we need to show if $\langle x , f ( x ) \rangle \rightarrow$ $\left\langle x ^ { * } , y ^ { * } \right\rangle$ then $\left\langle x ^ { * } , y ^ { * } \right\rangle \in f \left( x ^ { * } \right) .$ Let $x = ( \overline { p } , \overline { q } , \overline { r } )$ and $y = ( \overline { u } , \overline { v } , \overline { w } )$\\
 	$\left\langle x ^ { * } , y ^ { * } \right\rangle \in f \left( x ^ { * } \right) \Leftrightarrow y ^ { * } \in f \left( x ^ { * } \right) \Leftrightarrow$ \[\alpha \left( \overline { u } ^ { * } , \overline { q } ^ { * } , \overline { r } ^ { * } \right) \geq \alpha \left( \overline { p } ^ { \prime } , \overline { q } ^ { * } , \overline { r } ^ { * } \right) \forall \overline { p } ^ { \prime }\]
 	\[\beta \left( \overline { p } ^ { * } , \overline { v } ^ { * } , \overline { r } ^ { * } \right) \geq \beta \left( \overline { p } ^ { * } , \overline { q } ^ { \prime } , \overline { r } ^ { * } \right) \forall \overline { q } ^ { \prime }\]
 	\[	\gamma \left( \overline { p } ^ { * } , \overline { q } ^ { * } , \overline { w } ^ { * } \right) \geq \gamma \left( \overline { p } ^ { * } , \overline { q } ^ { * } , \overline { r } ^ { \prime } \right) \forall \overline { r } ^ { \prime }\]
$ 	\begin{array} { l } { \text { which shows } \{ \langle x , f ( x ) \rangle \} \text { is closed. And the above argument can easily apply to any finite number } } \\ { \text { players. } } \end{array} $\\
 	(Zhifeng Sun)\\
 	\\
 	This led to the creation of a famous Game Theory example called the "Prisoner's Dilemma" in 1950 by Merrill Flood and Melvin Dresher while working at RAND. The game goes as some version as such: \\
 		Two people are arrested and imprisoned for some sort of crime. Each prisoner is in solitary confinement and have no way of communicating with one another. There is insufficient evidence to convict the pair of the main charge, but enough to charge them of a lesser charge. Each prisoner is made an offer to either working against the other by betrayal and testifying the other committed the crime or to work together and remain silent. If both prisoners betray each other they will serve 3 years in jail. If one prisoner betrays and the other is silent, the betrayer is free to go but the silent prisoner is given a 5 year sentence. If both remain silent, they will be forced to serve 1 year sentences each.\\
 	This game makes the reader think about how for purely rational people who cannot communicate decide what would be the best way to proceed.
 	The Nash Equilibrium states that the best course of action is for both prisoners to betray each other. This is because betraying the other results in a higher potential reward and as there is no communication and trust between the two prisoners the other cannot know how the other will act. Therefore, it is the safest bet to simply betray. Similar concepts can be applied to the same basics as the Prisoner's Dilemma. Arm Races, cut-price marketing, environmental pollution are all examples that where working together would result in the best course for both players, but as trust and communication is minimal, mutual betrayal is the best solution for those involved.  
 	\section{Later Developments}
 	The work published in this time led to a rapid expansion in the 1950's on the ideas of Game Theory. Examples include: Fictitious play, extensive form game, repeated games, and the Shapley Value.  This time also saw an increase of more practical applications of Game Theory, being applied to philosophy and politics.
 	\subsection{Tit-for-tat}
 	 In 1979, Robert Axelrod was able to simulate players by using computer programs. The game was a tournament between programs, where the winner uses a tit-for-tat algorithm. The tit-for-tat program first cooperated with the opposing program, and from there on copied what the opponent program did the last round. Tit-for-tat could also be called equivalent retaliation as whatever the opponent did would be matched, weather good or bad. This result of a mostly cooperative strategy when facing opponents ends up explaining human nature quite well. Social psychologists and sociologist's study in tactics to reduce conflict. When two parties have been long in conflict with no trust, the best step in rebuilding trust is having one party cooperate. This will often cause the other party to cooperate in response. Over time this will rebuild trust and cooperation will be the default interaction. \\
 	 A variant of the prisoner's dilemma allows for infinitely repeated games. Using the tit-for-tat strategy results in further cooperation and higher payoff for the prisoner's. 
 	 \subsection{Nobel Work}
 	 Reinhard Selten, in 1965, was able to better define the Nash Equilibriam with the concept of the subgame perfect equilibria. Reinhard Selten, along with John Nash, became Economics Nobel Laureates in 1994 for their work.\\
 	 Thomas Schelling worked on dynamic models, which were the precursors to evolutionary Game Theory with help from Robert Aumann who worked on further defining and understanding equilibriums. They were awarded in 2005 as Nobel Laureates.\\
 	 Due to the work related to mechanism design theory, Leonid Hurwiz, Eric Maskin, and Rodger Myerson were all awarded the Nobel Prize in Economics in 2007.\\
 	 Even more recently was in 2012, with Alvin Roth and Lloyd Shapley. Followed by Jean Tirole in 2014.
 	\section{Types of Games}
 	In Game Theory, there are many different types of game that each serve different problems best. Initially there were only a limited number of game types, but as research has progress there is a greater range, variety, and uses for the differing game types.
 	\subsection{Cooperative and Non-Cooperative}
 		In Cooperative games, players best benefit through cooperation and most often binding agreements of cooperation are given and enforced. Players form groups called coalitions. Cooperative Game Theory looks more at the rewards and techniques that coalitions will use for optimization. In cooperative games there is no Nash equilibrium. In non-cooperative there are no binding rules which force players to cooperate. Instead, these are self enforced alliances where the main goal is to still maximize payout. In non-cooperative games, we are searching for a Nash equilibrium. 
 	\subsection{Symmetric and Asymmetric}
 		Symmetric games are the those that only depend on the strategy used and not who is playing. If the players can be interchanged, and the decisions remain the same the game is symmetric. The prisoner's dilemma is an example of symmetric game. No matter who is playing there is a clear strategy to be employed. For Asymmetric games, the players who are participating affect the results of the game. An example would be the entry of a new organization in a market as each organization will decide on differing methods of entering the market. One such game that is popular with economic experiments is called the ultimatum game. The game was first devised by Werner Guth, Rolf Schmittberger, and Bernd Schwarze.
 		\begin{quotation}
 			 The classical [Ultimatum] game involves two players who are given the opportunity to split \$10. One player proposes a potential split, and the other can accept, in which case the players receive the amounts in the proposal, or reject, in which case, both players receive nothing. (Chang, Levinboim, Maheswaran)
 		\end{quotation} 
 			 The Nash equilibrium entails having the first player offering the minimum amount of \$1 to the other. While keeping the maximum amount of \$9 for themselves. Logically the second player should accept, as \$1 is better than what they had before. 
 			 
 			 \begin{quote}
 			 	However, when experiments are conducted with human subjects, this behavior is rarely observed.  (Chang, Levinboim, Maheswaran)
 			\end{quote}
 		This is an interesting result to observe. As much as Game Theory prides itself in being logical and rational, humans are not completely logical and rational, and so in this case the expected results of the game differ from what actually occurs.
 	 \subsection{Evolutionary}
 	 	Evolutionary Game Theory is used to model evolving populations and how they change their strategies and react to change over time. This application of Game Theory is of interested for many fields including economists, sociologists, anthropologists, and philosophers. John Maynard Smith was the leading mathematical biologist who developed evolutionary Game Theory. His major interest being the discrepancy between Darwin's survival of the fitness and the sharing of resources so willingly in society.
 	\section{Applications}
 		The amazing aspect about Game Theory is the ability to be so apt applied to real life situations. As almost everything we do is in direct interactions with others, we can easily find examples where we are able to describe the possible outcomes and best solutions based on the "rules" of the given game.
 		\subsection{Politics}
 		In the early 60's there was the Cuban Missile Crisis, where the USSR tried to build a missile base in Cuba. The U.S.A set up a naval blockade to slow the supply ships. This issue can be simplified down to the game Chicken. Chicken involves two players running towards each other and seeing how close the players will get into running into each other. The winner is the player who stood their ground and didn't back off last second. Both users will lose if they run into each other. This is like the Cuban Missile Crisis as the first player to back down "loses", yet if no one backs down there would be a nuclear war of mutual assured destruction. This is a simplification of a much larger and complex issue, but the underlying idea is very similar. The ability to recognize what the most logical action to take will better our future.
 		
 		\subsection{Economy}
 		There have been numerous attempts to model the economy accurately using a variety of methods. By and large Game Theory has been used to model economic ideas and exemplify what will happen. While Game Theory can explain a lot and help us get a greater understanding for what we expect and what is actually happening, humans are not rational creatures and thus Game Theory cannot predict what doesn't make sense. Game theory has been great at modeling many many different economic topics from auction, bargaining, fair divisions, oligopolies, voting systems, information economics and many others. The extensive use of Game Theory for all these applications related to economics shows how much we serve to learn by further studying Game Theory.
 	\section{Future}
	Where Game Theory will take us next is an interesting question. As Game Theory is so apt to be applied to every situation faced by people, we can easily expect a greater use and study of Game Theory. Game Theory is consider a relatively new and understudied area of research. It will be interesting to see how Game Theorists better incorporate some of the irrationality that is so prevalent in people decisions. Several examples have shown even faced with a set of rational rules, many will go in face of what makes sense. Economics, politics, sociology, psychology, and many more areas have been greatly improved by the study of Game Theory. As we proceed into the future only more study will help us better describe the world around us.
	\pagebreak
	\section{Bibliography}
	Myerson, Roger B. Game Theory: Analysis of Conflict. Harvard University Press, 1997.\\
	\\
	Von Neumann, J., and Morgenstern, O., (1944). The Theory of Games and Economic 
	$  .\quad $	Behavior. Princeton: Princeton University Press.\\
	\\
	Cai, X., Liu, X. \& Qu, Z. J Supercomputer (2018). https://doi.org/10.1007/s11227-018-2628-7\\
	\\
	Chang YH., Levinboim T., Maheswaran R. (2012) The Social Ultimatum Game. In: Guy T.V., Kárný M., Wolpert D.H. (eds) Decision Making with Imperfect Decision Makers. Intelligent Systems Reference Library, vol 28. Springer, Berlin, Heidelberg
	
\end{document}


