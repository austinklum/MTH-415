\documentclass[12pt]{article}
\usepackage{color,latexsym,fancyhdr,amsmath,amsfonts,dsfont,amssymb,ragged2e}
\usepackage{color,soul}
\newtheorem{theorem}{Theorem}[section]


\newtheorem{claim}[theorem]{Claim}
\newcommand{\Z}{\mathds{Z}}
\newcommand{\R}{\mathds{R}}
\newcommand{\B}{\mathcal{B}}
\newcommand{\T}{\mathcal{T}}


\topmargin        -0.2 in
\textheight       8.4 in
\oddsidemargin    0 in  
\evensidemargin   0 in     
\textwidth        6.5 in
\headheight       15pt     
\headsep          .35 in     

\begin{document}
\pagestyle{fancy} \lhead{History of John Von Neumann} \chead{}
\rhead{\text{Austin Klum}} 
\lfoot{} \cfoot{} \rfoot{}

\begin{quotation}
	``In mathematics you don't understand things. You just get used to them.
	\begin{enumerate}
		\item[] ---John Von Neumann
	\end{enumerate}
\end{quotation} 

John Von Neumann was one of the most prolific mathematicians in the 20th century. His work provided us with many of the recent innovations of the past 100 years. Von Neumann filled many positions throughout his life from mathematician, physicist, computer scientist, chemist, military advisor, and many others roles Von Neumann was successful in most everything he did. We experienced a huge shift in knowledge and culture in the first half of the 20th century and Von Neumann was right there along with the changes. Without his work, we would be living in a much different world. His contributions to the pursuit of knowledge helped advance our society.

\section*{Early Life}
John Von Neumann was born in Hungary on December 28, 1903 and died in Washington D.C.  on February 8, 1957. His 53 year life span produced some of the greatest material of the 20th century. Von Neumann was a child prodigy who showed promise in languages, memorization, and mathematics. Von Neumann was born of an affluent family with his father, Max Neumann, being a top banker in Hungary. Max Neumann believed that knowledge of languages were important and had his children learn English, French, German, and Italian. In 1911, Von Neumann entered the Lutheran Fasori Evangélikus Gimnázium where he quickly excelled. The Luther Gymnasium was one of the best school for the elite in Budapest. While there he met and became friends with Eugene Wigner, a child who would one day also go on to perform groundbreaking work in Mathematics, Physics, and Engineering. As Von Neumann had shown such great mathematical aptitude, when he was 15 his parents had hired a private tutor named Gabor Szego. So impressed with the young prodigy,

\begin{quotation}
	``Mrs. Szego often recalled that Szego came home with tears in his eyes from his first encounter with the young prodigy. (Glimm, Impagliazzo, Singer)
\end{quotation}

In 1919, the Neumann family fled to Vienna to avoid the communist regime of Bela Kun. After completing secondary schooling in 1921 and deciding what to study further, Von Neumann's father discouraged him from pursuing a degree in Mathematics as Von Neumann's father did not see the financial payoff of Mathematics. Instead encouraging Von Neumann to pursue something that would be a good career path such as a chemical engineer. Compromising Von Neumann did both; earning a degree in Chemical Engineering in 1925 and a doctorate in Mathematics in 1926. One of his professors, George Pólya, went on to say about Von Neumann:

\begin{quotation}
	``Johnny was the only student I was ever afraid of. If in the course of a lecture I stated an unsolved problem, the chances were he'd come to me at the end of the lecture with the complete solution scribbled on a slip of paper. (Petkovic)
\end{quotation}

By the time, Von Neumann was 19 he had published two major math papers. One of which defined ordinal numbers as they are still used to this day. From 1926 to 1927 Von Neumann worked with David Hilbert, a famous German mathematician, on the goal of axiomatizing mathematics, which ultimately failed. In 1927, Von Neumann published 12 major papers in Mathematics. In 1928, Von Neumann started lecturing as the youngest adjunct professor ever at the University of Berlin. By the end of 1929, he had published 32 major papers. Von Neumann began to be known as an up and coming genius in the mathematical world. He was often pointed out at Mathematical conferences for his abilities. \\
He began traveling often to the United States in 1929 to lecture on quantum theory at Princeton University. He married his first wife Mariette Koevesi the following year. Once Adolf Hitler come to power, Von Neumann gave up his European academic roles and began to live in the United States full time.

\section*{Pure Mathematics}
	Von Neumann's innate ability to process and memorize information allowed him to make many advances in multiple fields of study. He was the founder of many fields and for field he dabbled in many of his contemporaries commented on how he was able to rapidly understand and come up with new results. Often times his contribution to each field would be comparable to a milestone of achievement for a standard researcher cumulating over a lifetime of study on the subject. Yet Neumann would only spend a fraction of the time on the field and still make great leaps and bound in discover. David Blackwell, an American statistician and mathematician who contributed to Game Theory, went on record in 1986 about the time he spoke to Von Neumann about his recent thesis:
	\begin{quotation}
		He listened for about ten minutes and asked me a couple questions, and then stated \textit{me} about \textit{my} thesis. What you really done is this, and probably this is true, and you could have done it in a somewhat simpler way, and so on. He was a really remarkable man. He listened to me talk about this rather obscure subject and in ten minutes he knew more about it than I did. (Schmalz)
	\end{quotation}
  This ability to unbelievably quickly and throughly synthesize information is a common theme throughout Von Neumann's life.
\subsection*{Quantum Mechanics}
 	Von Neumann was the first to form rigorous mathematical frameworks for Quantum Mechanics. He also was able to relate much of quantum mechanics to work he had previously encounter. He viewed many of the quantum states as vectors in a Hilbert Space. From this, Von Neumann was able to prove that deterministic "hidden variables" could not be a plausible explanation. This furthered the idea of indeterminacy of quantum theory. Stating something cannot be determined until it is observed. Similar to Schrödinger's cat where the cat is both dead and alive until observation or light photons where light is both wave and a particle until observed. While many were excited from this result, others, namely Albert Einstein, were dismayed by the result of indeterminacy.
\subsection*{Game Theory}
	\begin{quotation}
				``Game Theory can be defined as the study of mathematical models of
		conflict and cooperation between intelligent rational decision-makers.
		Game Theory provides general mathematical techniques for analyzing
		situations in which two or more individuals make decisions that will
		influence one another's welfare. (Myerson, R.B. 1)
	\end{quotation}

	
	Von Neumann was one of the found fathers of Game Theory. In 1928, he proved his minmax theorem which states that in games where each player's gain or loss is balanced by the loss or gain by the other player, called zero-sum games, with both players knowing all moves that have taken place thus far, known as perfect information, there exists a pair of strategies that allows both players to minimize their maximum loss. This minimizing maximum loss gives us the name minmax. He helped further define the field over time. He went on to write more in his 1944 book \textit{Theory of Games and Economic Behavior} coauthored with Oskar Morgenstern. Game Theory has many interesting applications and uses. It will surely by used to a greater extent as time goes on.
\section*{Applied Mathematics}
\subsection*{Nuclear Weapons}
	Shortly after being naturalized as a citizen of the United States in 1937, Von Neumann began working as a consult for the United States military. This eventually led to his involvement in the Manhattan Project. This project was a research and development effort during World War II that resulted in the first nuclear weapons. Von Neumann's biggest contribution to the project was the ideas and design of the explosive lenses that were used to compress the plutonium core. 
	He was a strong support of an implosion concept used in the detonation of the weapons. After some initial disagreement from colleagues and the lack of the correct uranium use for other methods, implosion was the method used in the bombings of Nagasaki and Hiroshima. He also went on to show that if the atomic bomb would be denoted above the target as opposed to ground level then the shock wave was greater and therefore more effective. Von Neumann with four scientists and many other military personnel were selected as the committee to decide which Japanese cities should be the first to be decimated. Kyoto was Von Neumann's first choice as opposed to the actually cities that were targeted.
	After the bombings of Nagasaki and Hiroshima, Von Neumann along with Edward Teller and Klaus Fuchs continued their work on the hydrogen bomb project. They went on to help design a new generations of nuclear weapons even more powerful than the first. \\
	\\
	Von Neumann also developed the equilibrium strategy of mutual assured destruction. No doubt influences from his ideas in game theory had a hand in the development of this strategy. He strongly pushed for the implementation of mutual assured destruction as he feared the USSR was doing similar work. If the United States did not have an equally large arsenal of nuclear weapons, then the USSR would be able to attack and take over the world.
\subsection*{Computers}
Von Neumann was also one of the founding fathers in computing. He was the first to come up with the Merge Sort Algorithm in 1945. The Merge Sort Algorithm is the idea of splitting an array recursively, sorting the smaller parts of the array, and returning the sorted resulting array. Von Neumann used computers in his work relating to the hydrogen bomb. He also had a hand in the development of the Monte Carlo method, using randomness to determine results. Von Neumann notice that computing truly random numbers was painfully slow and so was one of the first to develop pseduorandom numbers. While not perfect, the method was fast and worked well enough for Von Neumann's uses. 

\begin{quotation}
	``Anyone who considers arithmetical methods of producing random digits is, of course, in a state of sin. For as has been pointed out several times before there is no such thing as a random number --- there are only methods to produce random numbers and a strict arithmetic procedure of course is not such a method. (Von Neumann)
\end{quotation}

He also wrote about in his \textit{First Draft of a Report on the EDVAC} of a computer architecture that stored both memory and instructions in the same address space. This architecture is still used today in modern computer deigns. This computer architecture is now known as the Von Neumann Architecture.\\
\\
In Von Neumann's unfinished book \textit{The Computer and the Brain}, he discusses the similarities between the human brain and computing machines. In the book he goes on to discuss several close similarities between the two and how they differ. One particular section discusses the use of memory and totally memory capacity.
\begin{quotation}
	The memory capacity required for a modern computing machine is usually of the order of 105 to 106 bits. The memory capacities to be surmised as necessary for the functioning of the nervous system would seem to have to be a good deal larger than this, since the nervous system as such was seen above to be a considerably larger automaton than the artificial automata (e.g. computing machines)	that we know. (The Computer and the Brain)
\end{quotation}
	Upon pondering this question further and giving rough estimates and some calculations, Von Neumann came up with an estimate memory capacity at $ 2.8 \times 10^{20} $ bits. That is around $ 35,000 $ Petabytes of data where 1 Petabyte gives us roughly $ 1,000,000 $ Gigabytes. This pales in comparison to the Internet which was estimated in 2016 by Cisco to be around $ 1,000,000 $ Petabytes (Cisco)
\subsection*{Weather and Climate}
	Von Neumann and his team were some of the first to numerically forecast the weather. Founded in 1946 at Princeton, the "Meteorological Program" with funding from the United Sates Navy began working on modeling climate and weather using the early computers of the time. In 1950, Von Neumann and team published their work as \textit{Numerical Integration of the Barotropic Vorticity Equation}. They were the trail blazers looking at the role of sea-air exchange on its effects on climate.\\
	\\
	From this research Von Neumann was able to deduce that human activity can have a change on climate and most likely already has. Von Neumann went on to say in 1955:
	\begin{quotation}
		``Carbon dioxide released into the atmosphere by industry's burning of coal and oil - more than half of it during the last generation - may have changed the atmosphere's composition sufficiently to account for a general warming of the world by about one degree Fahrenheit. (Rosenthal)
	\end{quotation}
	He was also extremely cautious about intentionally trying to change climate whether it be in response to the accidental climate change we are currently doing or otherwise.
	\begin{quotation}
		``What could be done, of course, is no index to what \textit{should} be done... (Rosenthal)
	\end{quotation}
	Going on to discuss all complications and intricacies that are presented in such a problem as changing the climate. He was concerned whether we truly would understand the system well enough to cause more good than harm.
	\begin{quotation}
		``But there is little doubt that one could carry out the necessary analyses needed to predict the results, intervene on any desired scale, and ultimately achieve rather fantastic results. (Rosenthal)
	\end{quotation}

\section*{End of Life}
In 1955, Von Neumann was diagnosed with prostate cancer. Perhaps one of the saddest ends of a great mind, Von Neumann struggled with the thought of death. He was afraid of no longer existing and vehemently opposed his demise. His impending death frustrated him for throughout his entire life he had been able to use his mind to solve his problems. But this problem he could not be solved. Some closing statements given by others on Von Neumann's handle of this early end of life.
\begin{quotation}
	Eugene Wigner wrote:\\
	When Von Neumann realised he was incurably ill, his logic forced
	him to realise that he would cease to exist, and hence cease to have
	thoughts... It was heartbreaking to watch the frustration of his mind,
	when all hope was gone, in its struggle with the fate which appeared
	to him unavoidable but unacceptable.\\
	\\
	His biographer, S.J. Heims, would add the following:\\
	\dots. his mind, the amulet on which he had always been able to rely was becoming less dependable. Then came complete psychological breakdown; panic, screams of uncontrollable terror every night. His friend Edward Teller said, ``I think that Von Neumann suffered more when his mind would no longer function, than I have ever seen any human being suffer." (Read)
\end{quotation}
Shortly before his death, Von Neumann sent for a priest to visit him. For much of his life, Von Neumann, was agnostic. But as death loomed, with the influence of the Pascal Wager stating that humans bet with their lives that there either is or isn't a god, Von Neumann reportedly said
\begin{quotation}
	 ``So long as there is the possibility of eternal damnation for nonbelievers it is more logical to be a believer at the end.
\end{quotation}
Even after conversion to Catholicism, Von Neumann did not find much comfort and remained terrified of death.
\section*{Conclusion}
	Von Neumann left behind a legacy. He was able to cross between multiple fields and link them together. His quick mind was his most valuable asset. He was able to easily understand and process information. This allowed him to make so many ground breaking discoveries and ideas. Without his influence, much of the modern amenities we have today would not have been possible. It is greatly disappointing to see his life end so early, for even up to his death he worked passionately on so many different subjects. One can only imagine what the future could have been if Von Neumann was able to live to a much older age. By far one of the most important mathematicians of the 20th century, his legacy and prowess will be forever known.
	
\pagebreak	
	
\section*{References}
Glimm, James; Impagliazzo, John; Singer, Isadore Manuel (1990). The Legacy of John von Neumann. American Mathematical Society. ISBN 978-0-8218-4219-5.\\
\\
Petković, Miodrag (2009). Famous puzzles of great mathematicians. American Mathematical Society. p. 157. ISBN 978-0-8218-4814-2.\\
\\
Myerson, Roger B. Game Theory: Analysis of Conflict. Harvard University Press, 1997.\\
\\
Von Neumann, John (1951). "Various techniques used in connection with random digits" (PDF). National Bureau of Standards Applied Mathematics Series. 12: 36–38.\\
\\
Engineering: Its Role and Function in Human Society edited by William H. Davenport, Daniel I. Rosenthal (Elsevier 2016), page 266\\
\\
Von Neumann, John , et al. The Computer and the Brain. Yale University, 1958.\\
\\
“The Zettabyte Era Officially Begins (How Much Is That?).” Blogs@Cisco - Cisco Blogs, 11 Oct. 2016, blogs.cisco.com/sp/the-zettabyte-era-officially-begins-how-much-is-that.
\\
Schmalz, Rosemary, et al. Out of the Mouths of Mathematicians: a Quotation Book for Philomaths. Mathematical Association of America, 1993.\\
\\
Read, Colin (2012). The Portfolio Theorists: von Neumann, Savage, Arrow and Markowitz. Great Minds in Finance. Palgrave Macmillan. p. 65. ISBN 978-0230274143. \\


	                              
\end{document}


