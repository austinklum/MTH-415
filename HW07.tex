\documentclass[12pt]{article}
\usepackage{color,latexsym,fancyhdr,amsmath,amsfonts,dsfont,amssymb}
\usepackage{color,soul}
\newtheorem{theorem}{Theorem}[section]


\newtheorem{claim}[theorem]{Claim}
\newcommand{\Z}{\mathbb{Z}}
\newcommand{\R}{\mathbb{R}}
\newcommand{\Q}{\mathbb{Q}}
\newcommand{\B}{\mathcal{B}}
\newcommand{\T}{\mathcal{T}}


\topmargin        -0.2 in
\textheight       8.4 in
\oddsidemargin    0 in  
\evensidemargin   0 in     
\textwidth        6.5 in
\headheight       15pt     
\headsep          .35 in     


\begin{document}
	\pagestyle{fancy} \lhead{MTH 415 Homework 07} 
	\chead{04/12/2019}
	\rhead{Austin Klum}
	\lfoot{} \cfoot{} \rfoot{}
	
	\begin{enumerate}
		\item[6.01] Prove that an infinite set with the finite complement topology is a connected
		topological space.\\
		Assume $ X $ with the finite complement topology is the union of two disjoint closed subsets of $ X $, namely $ U,V $. That is $ X= U\cup V $. Since, $ U$ and $V $ are closed they are finite by definition. This is a contradiction as $ X $ is the infinite sets and two finite sets do not equal the infinite set.
		
		\item[6.02] \textbf{Prove Theorem $6.2 :$} A topological space $X$ is connected if and only if there are no nonempty proper subsets of $X$ that are both open and closed in $X .$\\
		Suppose that $ X $ is not connected. Let $ U,V $ be a separation of $ X $. That is $ U\cup V = X$ and $ U,V $ are disjoint nonempty open sets. Notice, $ V=X-U  $ as $ U $ and $ V $ are disjoint. Then, we have that $ U $ is closed as it is the complement of an open set. We then have $ V=X-U\not= \varnothing $. This results says that $ U $ is a proper subset. Thus, $ U $ is a nonempty, proper, closed, and open set.\\
		\\
		Suppose $ U$ is a nonempty proper set that is closed and open. Let $ V=X-U $. Since, $ U $ is proper we have $ V $ is nonempty and open. Notice.
			\[X=U\cup(X-U)=U\cup V\]
		Thus, $ U,V $ form a separation of $ X $.\\
		Thus, $ X $ is not connected.\\
		\\
		Therefore, $X$ is connected if and only if there are no nonempty proper subsets of $X$ that are both open and closed in $X .$\\
		\item[6.07] \begin{enumerate}
			\item[(a)]  Prove that if a topological space $X$ has the discrete topology, then $X$ is totally disconnected.\\
			Assume $ X $ has the discrete topology and $ A\subset X $ contains more than one point. Let $ x',x\in A $ with $ U=\{x'\} $ and  $V = A-\{x\} $. Hence, $ V $ is nonempty as $ A $ was defined with more than one point. Since, singletons are open and the union of singletons are open in the discrete topology, we must have that $ U $ and $ V $ are open. Then, $ U\cap V = \varnothing $ and $ U\cup V = A $. \\
			Thus, $ U $ and $ V $ form a separation of $ A $.\\
			Hence, $ A $ is disconnected\\
			So, only one point sets are connected\\
			Thus, $ X $ is totally disconnected\\
			Therefore, if a topological space $X$ has the discrete topology, then $X$ is totally disconnected
			\item[(b)] Let $\mathbb { Q }$ be the set of rational numbers with the standard topology. Prove	that $\mathbb { Q }$ is totally disconnected. (This exercise and Example 6.9 demonstrate that the converse to the result in part (a) does not hold. In both cases, the space is totally disconnected but does not have the discrete topology.)\\
			Let $ x,y \in \mathbb{ Q } $ such that $ x\not=y $. Without loss of generality assume $ x<y $. Notice, there exists an irrational number $ z $ such that $ x<z<y $. Then, $ x $ and $ y $ are in different components of $ \mathbb{ Q } $. Hence, the components of $ \mathbb{ Q } $ are singletons.\\
			Thus, $ \mathbb{ Q } $ is totally disconnected.
		\end{enumerate}
		
		\item[6.09] \textbf{Prove Theorem 6.12, parts(ii) and (iii):} Let $ X $ be a topological space.
		\begin{enumerate}
			\item[(a)]If $A$ is connected in $X ,$ then $A$ is a subset of a component of $X .$\\
				Suppose $ A $ is connected in $ X $. Notice, there exists $ \sim_c $ defined by $ x \sim_c y $ for  $ x $ and $ y $ that lie in $A$. This forms a component of $ X $.\\
				Therefore, $ A $ is a subset of a component of $ X $.
			\item[(b)]Each component of $X$ is a closed subset of $X .$\\
				Let $ C $ be a component of $ X $. We then know that $ C $ is connected in $ X $. Suppose $ x \in X $ and $ x\notin C $. Define $ B = C\cup \{x\} $, which is not connected. Let $ U,V $ form a separation of $ B $ and $ x\in U $. That is $ B=U\cup V $. Notice, $ U\cup C = \varnothing $ as if it weren't true $ U\cap C $ and $ V\cap C $ would form a separation of $ C $. But $ C $ is connected and thus cannot have a separation. \\
				Thus, $ C $  has an open complement\\
				Therefore, $ C $ is Closed 
			\item[(c)] Provide an example showing that the components of $X$ are not necessarily
			open subsets of $X .$\\
			Consider $ \mathbb{ Q } $. Notice components of $ \mathbb{ Q } $ are singletons. Hence, components of $ \Q $ are not open.
		\end{enumerate}
		
		\item[6.10] The following examples demonstrate that the condition $U \cap V \cap A = \varnothing$ is appropriate in the definition of a separation of $A$ in $X$ and that the condition would be too strong if it required $U \cap V = \varnothing$ :
		\begin{enumerate}
			\item[(a)] Find an example of a topology on $X = \{ a , b , c \}$ and a disconnected subset
			$A$ such that every pair of sets, $U$ and $V ,$ that is a separation of $A$ in $X$
			satisfies $U \cap V \cap ( X - A ) \neq \varnothing$ .\\
			Using the discrete topology, let $ A = \{a,b\} $. Notice, $ X-A =\{c\} $. Then, an example that satisfies this is $ U = \{a,c\} $ and $ V = \{b,c\} $. As $ U,V $ form a separation on $ A $ and satisfy $U \cap V \cap ( X - A ) \neq \varnothing$. 
			\item[(b)] Find a topology on $\mathbb { R }$ and a disconnected subset $A$ such that every pair of
			sets, $U$ and $V ,$ that is a separation of $A$ in $\mathbb { R }$ satisfies $U \cap V \cap ( \mathbb { R } - A ) \neq \varnothing$
		\end{enumerate}
		\item[6.18] Give examples of subsets $A$ and $B$ in $\mathbb { R } ^ { 2 }$ such that
		\begin{enumerate}
			\item[(a)] $A$ and $B$ are connected, but $A \cap B$ is not.\\
			 Let $ A = \{B((0,0),1)\} $ and $ B=\{B((10,0),1)\} $. Notice, $ A $ and $ B $ are connected, but $ A\cap B $ is not connected.
			\item[(b)] $A$ and $B$ are connected, but $A - B$ is not.\\
				Let $ A = \{(x,0)\in \R^2|0\leq x\leq1\} $ and $ B=\{(x,0)\in \R^2|0<x<1\} $. Then, $ A-B= \{(0,0)\}\cup\{(1,0)\} $
			\item[(c)] $A$ is connected, $B$ is disconnected, and $A \cap B$ is connected.
				Let $A= \{B((0,0),1)\}$ and $ B=A\cup\{B(10,0),1)\} $. Notice, $ A $ is connected and $ B $ is disconnected. Then, $ A\cap B = A$.\\
				 Thus, $ A\cap B $ is continuous as $ A $ is continuous.
			\item[(d)] $A$ and $B$ are disconnected, but $A \cup B$ is connected.\\
			Let $ A=\{[0,1]\cup[2,3]\} $ and $ B=\{[1,2]\cup[3,4]\} $. Notice, $ A $ and $ B $ are disconnected, but $ A\cup B = \{(x,0)\in\R^2|0\leq x \leq 4\} $ is connected.
			\item[(e)] $A$ and $B$ are connected, $\mathrm { Cl } ( A ) \cap \mathrm { Cl } ( B ) \neq \varnothing ,$ and $A \cup B$ is disconnected.
		\end{enumerate}
		
		\item[6.20] In each of the following cases, prove whether or not the given set $C$ is a cutset of the connected topological space $X :$
		\begin{enumerate}
			\item[(a)] $C = \{ b \}$ and $X = \{ a , b , c \}$ with topology $\{ \varnothing , \{ b \} , \{ a , b \} , \{ b , c \} , X \}$\\
			$ X - \{b\} = \{a,c\}$ and the topology is $\{\varnothing,\{a\},\{c\},X-\{b\}\} $. Thus, disconnected.\\
			$ C $ is a cutset
			\item[(b)] $C = \{ c \}$ and $X$ is the same as in (a).\\
			$ X-\{c\}=\{a,b\} $ and the topology is $ \{\varnothing, \{b\},\{a,b\},X-\{c\}\} $. Thus, connected.\\
			$ C $ is not a cutset.
			\item[\textcolor{red}{(c)}] $C = \{ 0 \}$ and $X = P P \mathbb { R } _ { 0 } ,$ the particular point topology on $\mathbb { R }$ with the origin as the particular point.\\
			$ \R  -\{0\} = (-\infty,0)\cup(0,\infty)$ which is closed. Thus, we cannot form two disjoint open nonempty sets such that their union is $ X $.\\ 
			$ C $ is not a cutset.
			\item[(d)]	$C = \{ - 1,1 \}$ and $X = \mathbb { R } _ { f c } ,$ the real line in the finite complement topology.\\
			As we are on the real line, if we cut a point out we are no longer connected.
			$ C $ is a cutset.
			\item[(e)] $C = $ the equator in $ S^2 $. \\
			When you cut the equator of the sphere $ S^2 $, you end up with two disjoint sets $ NH $ and $ SH $ where $ NH\cup SH = S^2$. Thus, disconnected.\\
			$ C $ is a cutset.
			\item[(f)] $C =$ the core curve in the Mobius band $X ,$ as shown in Figure 6.16 .\\
			As the twist in the band allows us to get to either side of the core curve, we are still connected. (Not really sure how to describe this. I made a Mobius band and I could get around to all elements.)\\
			$ C $ is not a cutset.
		\end{enumerate}
		
		\item[6.22] Can you cut a Klein bottle into two Mobius bands? Find a cutset $C$ for the
		Klein bottle $K$ such that
		 \begin{enumerate}
		 	\item[(i)]  $C$ is a simple closed curve in $K ,$ and
		 	\item[(ii)] $K - C$ is a union of two disjoint open sets such that the closure of
		 	each in $K$ is homeomorphic to a Mobius band.
		 \end{enumerate}
	 	Yes, using the core curve of the Klein bottle we result in two Mobius bands that meet the requirements. (You cut the Klein bottle in half and you'll have two Mobius bands)
		\item[6.24] Prove Theorem $6.20 :$ Let $f : X \rightarrow Y$ be a homeomorphism. If $S$ is a cutset of $X$ , then $f ( S )$ is a cutset of $Y .$
			Suppose $ f: X \rightarrow Y $ is a homeomorphism and $ S $ is a cutset of $X$. Notice, $ X-S $ is disconnected. (WTS: $ Y-f(S) $ is disconnected).\\
			 We can then restrict the domain of $ f $ to $ X-S $. Define $ f':X-S \rightarrow Y-f'(S) $. Note, $ f' $ is a homeomorphism since $ f $ is a bijection and restriction of continuous maps are continuous. Since, $ f' $ is a homeomorphism and $ X-S $ is disconnected, $ Y-f'(S) $ must also be disconnected.\\
			 Therefore, $ f'(S) $ is a cutset of $ Y $
		\item[\textcolor{red}{6.26}] Prove that for every $n \geq 2$ neither the line nor the circle is homeomorphic to	$S ^ { n } .$\\
		Suppose not. That is suppose the line and the circle are homeomorphic to $ S^n $. Take a cutpoint in the line or the circle. We know that a singular point in is a cutset on the line and the circle. This does not hold for $ S^n $. This is a contradiction as $ S^n $ doesn't have a cutpoint but rather a cutset. \\
		Therefore the line and the circle are not homeomorphic to $ S^n $.
		
		\item[6.32] Let $T : S ^ { 2 } \rightarrow \mathbb { R }$ be defined by equating the sphere with the surface of the Earth
		and letting $T ( x )$ be the temperature at point $x$ on the surface at some given time.
		Assume that $T$ is a continuous function. Show that if $T ($ Anchorage $) = - 30 ^ { \circ }$ and $T ($ Honolulu $) = 80 ^ { \circ }$ , there is some point on the Earth where the temperature	is $0 ^ { \circ } .$ Does the conclusion necessarily hold if we restrict the domain to the fifty United States? Does the conclusion hold in the United States if $T$ (Duluth) $=- 30 ^ { \circ }$ and $T ($ Fort Lauderdale $) = 80 ^ { \circ } ?$\\
		
		No, as the fifty United States is disconnected and so the intermediate value theorem does not hold.\\
		Yes, as the continental USA is connected, so the intermediate value theorem hold. Thus, there is a point where the temperature is $ 0^\circ $. 
		
		\item[6.33] Let $p ( x )$ be an odd-degree polynomial function. Prove that $p ( x ) = 0$ has at least one real solution.\\
		Let $ p(x) $ be an odd-degree polynomial function. Notice, $ p $ is continuous as polynomial functions are continuous by definition. Suppose $ p(a') $ is some positive value $ a $ and $ p(b') = b $ is some negative value. Then by the intermediate value theorem we must that $ f(x)=0 $ for some $ x $.\\
		\textcolor{red}{I never used anything about it being an odd-degree poloynomial. Also did a lot of assuming to make this work...}
		
		\item[6.36] Suppose that at a given time we measure the intensity of sunlight at each point on the Earth's surface. According to Theorem 6.26 , there must be a pair of points
		opposite each other on the Earth's surface at which the intensity of sunlight is the same. However, if it is daytime at one point, it must be nighttime at the
		point opposite it! Resolve the paradox.\\
		The intensity of sunlight is matched on the opposing sides of Earth's surface. For example when a certain place is experiencing dawn and another place is experiencing twilight, the two places have the same intensity of sunlight. Thus, this the paradox is resolved as there are a pair of points opposite each other on the Earth's surface at which the intensity of sunlight is the same.
		
		\item[\textcolor{blue}{6.38}] \begin{enumerate}
			\item[(a)] Let $X$ be connected, and assume a homeomorphism $A : X \rightarrow X$ exists
			such that $A \circ A ( x ) = x$ for all $x \in X .$ Prove that, for every continuous
			function $f : X \rightarrow \mathbb { R }$ , there exists $x \in X$ such that $f ( x ) = f ( A ( x ) )$ .\\
			Suppose $ A:X\rightarrow X $ is a a homeomorphism such that $ \forall x \in X, A\circ A(x) =x$ and $ f:X\rightarrow \R $ is continuous. Let $ x\in X $. Notice, $ A(A(x))=x $. So, $ A $ is its own inverse. Hence, $ A(x)=x $.\\
			Thus, $ f(x)=f(A(x)) $\\
			Therefore, $ \exists x\in X $ such that $ f(x)=f(A(x)) $ 
			\item[(b)] Use the result from part (a) to prove that somewhere on a glazed doughnut
			there is a point that has the same thickness of glazing as the point obtained
			by 180 -degree rotation through the central axis of the doughnut. \\
			Let $ X $ be our doughnut, our continuous function $ f $ be the thickness of the glaze, and $ A: X \rightarrow X $ be our rotation of the doughnut. By part (a), we have that there must exist a point $ x $ such that $ f(x)=f(A(x)) $. 
		\end{enumerate}
		
		\item[6.39] Let $X = \{ a , b \}$ and assume that $X$ has the topology $\mathcal { T } = \{ \varnothing , \{ a \} , X \} .$ Show that $X$ is path connected in this topology.\\
		We define a function $ f:[0,1]\rightarrow X $ with $ f(t)=(1-t)a+tb $. Notice, $ f(0)=a $ and $ f(1)=b $. $ f $ must also be continuous by definition. Thus, there is a path between any two point in $ X $.
		\textcolor{red}{Is this enough? Do I need to say something about the topology itself?}
		
		\item[6.41] A set $C \subset \mathbb { R } ^ { n }$ is said to be star convex if there exists a point $p ^ { * } \in C$ such that
		for every $p \in C ,$ the line segment in $\mathbb { R } ^ { n }$ joining $p ^ { * }$ and $p$ lies in $C .$ Prove that
		if $C \subset \mathbb { R } ^ { n }$ is star convex, then $C$ is path connected in $\mathbb { R } ^ { n } .$\\
		Let $ C $ be a start convex. Then we must have a $ p*\in C $ and $ p\in C $ such that the line segment in $ \R^n $ joining $ p^* $ and $ p $ lies in $ C $. Thus, there exists a path for every two points in $ C $. Therefore, $ C $ is path connected.\\
		\textcolor{red}{Look this one over again. Seems too easy and must be missing something in my understanding of the question.}
		
		\item[6.45] Prove that the $n$ -sphere $S ^ { n }$ is path connected for $n \geq 1 .$ (Hint: Find a surjective continuous function from a path connected space to $S ^ { n })$ \\
		Let $ f: \R^{n+1}-\{0\} \rightarrow S^n$ with $ f(x)=\frac{x}{|x|} $. Notice, as 0 is not in the domain, we must have that $ f $ is continuous. We define $ S^n = \{x\in\R^{n+1}| |x| = 1\} $. Then, $ f $ is surjective as all of $ S^n $ must be contained by $ \R^{n+1} -\{0\}$. Thus, suppose $ x \in S^n $, we then have $ f(x)=\frac{x}{1} = x $. Thus, $ f(x)\in S^n $\\
		Since, $ f $ is continuous and $ \R^{n+1}-\{0\} $ is path connected, we then have $ f(\R^{n+1}-\{0\}) $ is a path connected subspace of $ S^n$. Note, as $ f $ is also a surjection we know $ f(\R^{n+1}-\{0\}) =  S^n $\\
		Therefore, $ n $-sphere $ S^n{0} $ is path connected.
		\item[6.51] Prove that if $f : X \rightarrow Y$ is a homeomorphism and $C$ is a path component of $X ,$ then $f ( C )$ is a path component of $Y .$
			Let $ f:X\rightarrow Y $ be a homeomorphism and $ C $ be a path component of $ X $. Let $x_1,x_2\in X$ such that $ x_2 \in C$ and $ x_1 \sim_x x_2 $. Notice, there must exist a continuous function $ g:[0,1]\rightarrow X $ such that $ g(0)= x_1$ and $ g(1)=x_2 $. Since the composition of continuous functions is continuous we have $ f\circ g:[0,1]\rightarrow Y $ is also a continuous function. Let $g' = f\circ g  $ and $ y_1 = f(x_1) $ and $ y_2=f(x_2) $. Observe.
				\[g'(0)=f\circ g(0)=f(x_1)=y_1\]
				\[g'(1)=f\circ g(1)=f(x_2)=y_2\]
			Thus, $ g' $ maps paths from $ f(C) $ to $ Y $. \\
			Therefore, $ f(c) $ is a path component of $ Y $.
	\end{enumerate}
	\section*{Summary}
	I am getting a better handle on the material in class. I have been spending significant more time going to office hours and working on the homeworks over multiple sessions. I'm still having a difficult time, but the situation does not seem entirely hopeless now. These next 4 weeks are going to be grueling and long, but will be very rewarding once completed. I plan on continuing to come to office hours often and working on the homework well in advance so I can ask better questions.
\end{document}


