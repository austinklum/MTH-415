\documentclass[12pt]{article}
\usepackage{color,latexsym,fancyhdr,amsmath,amsfonts,dsfont,amssymb}
\usepackage{color,soul}

\newtheorem{theorem}{Theorem}[section]
\newtheorem{claim}[theorem]{Claim}
\newcommand{\Z}{\mathds{Z}}
\newcommand{\R}{\mathds{R}}
\newcommand{\B}{\mathcal{B}}
\newcommand{\T}{\mathcal{T}}
\newcommand{\w}[1]{\textcolor{red}{#1}}

\newenvironment{proofed}[1][]{\par \medskip \noindent \textbf{#1 Proof: }}{\hfill$\square$}
\newenvironment{defn}[2][]{\par \medskip \noindent \textbf{#1 Definition of \large#2 \medskip \\}}{\rmfamily \medskip}
\newenvironment{thm}[2][]{\par \medskip \noindent \textbf{#1 Theorem about \large#2 \medskip \\}}{\rmfamily \medskip}


\topmargin        -0.2 in
\textheight       8.4 in
\oddsidemargin    0 in  
\evensidemargin   0 in     
\textwidth        6.5 in
\headheight       15pt     
\headsep          .35 in     


\begin{document}
\pagestyle{fancy} \lhead{MTH 415 Study Guide} \chead{Exam 2}
\rhead{Austin Klum} 
\lfoot{} \cfoot{} \rfoot{}

\setcounter{section}{3}

\section{Continuous Functions and Homeomorphisms}
	\subsection{Definition of Continuous}
		A function $f : \mathbb { R } \rightarrow \mathbb { R }$ is \textbf{continuous} if for every $x _ { 0 } \in \mathbb { R }$ and every $\varepsilon > 0 ,$ there exists $a \delta > 0$ such that if $\left| x - x _ { 0 } \right| < \delta ,$ then $\left| f ( x ) - f \left( x _ { 0 } \right) \right| < \varepsilon$
		
	\subsection{Open Set Definition of Continuity}
		Let $X$ and $Y$ be topological spaces. A function $f : X \rightarrow Y$ is \textbf{continuous} if $f ^ { - 1 } ( V )$ is open in $X$ for every open set $V$ in $Y .$\\
		\\
		We call this the \textbf{open set definition of continuity}. Paraphrased, it states
		that $f$ is continuous if the preimage of every open set is open.
	
	\subsection{Theorem that a function is continuous if and only if the preimage of the basis elements is open}
		Let $X$ and $Y$ be topological spaces and $\mathcal { B }$ be a basis for the
		topology on $Y$ . Then $f : X \rightarrow Y$ is continuous if and only if $f ^ { - 1 } ( B )$ is open
		in $X$ for every $B \in \mathcal { B }$ .
	\begin{proofed}
			Suppose $f : X \rightarrow Y$ is continuous. Then $f ^ { - 1 } ( V )$ is open in
		$X$ for every $V$ open in $Y .$ Since every basis element $B$ is open in $Y ,$ it
		follows that $f ^ { - 1 } ( B )$ is open in $X$ for all $B \in \mathcal { B }$ .\\
		\\
		Now, suppose $f ^ { - 1 } ( B )$ is open in $X$ for every $B \in \mathcal { B } .$ We show that
		$f$ is continuous. Let $V$ be an open set in $Y$ . Then $V$ is a union of basis
		elements, say $V = \cup B _ { \alpha }$ . Thus,
			\[f ^ { - 1 } ( V ) = f ^ { - 1 } \left( \cup B _ { \alpha } \right) = \cup f ^ { - 1 } \left( B _ { \alpha } \right)\]
		By assumption, each set $f ^ { - 1 } \left( B _ { \alpha } \right)$ is open in $X ;$ therefore so is their
		union. Thus, $f ^ { - 1 } ( V )$ is open in $X$ , and it follows that the preimage of
		every open set in $Y$ is open in $X$ . Hence, $f$ is continuous.
	\end{proofed}

		
	\subsection{Theorem that every polynomial is continuous}
		Let $\mathbb { R }$ have the standard topology. Then every polynomial
		function $p : \mathbb { R } \rightarrow \mathbb { R } ,$ with $p ( x ) = a _ { n } x ^ { n } + \ldots + a _ { 1 } x + a _ { 0 } ,$ is continuous.
		
	\subsection{Theorem that says the closure of a subset maps to part of the closure of the superset}
		Let $f : X \rightarrow Y$ be continuous and assume that $A \subset X .$ If
		$x \in C l ( A ) ,$ then $f ( x ) \in C l ( f ( A ) )$.
	\begin{proofed}
		Suppose that $f : X \rightarrow Y$ is continuous, $x \in X ,$ and $A \subset X$ .
		We prove that if $f ( x ) \notin \operatorname { Cl } ( f ( A ) ) ,$ then $x \notin C l ( A ) .$ Thus suppose that $f ( x ) \notin \operatorname { Cl } ( f ( A ) ) .$ By Theorem 2.5 there exists an open set $U$ containing
		$f ( x ) ,$ but not intersecting $f ( A ) .$ It follows that $f ^ { - 1 } ( U )$ is an open set
		containing $x$ that does not intersect $A$ . Thus $x \notin C I ( A ) ,$ and the result
		follows.
	\end{proofed}
		
	\subsection{Translation of $ \varepsilon - \delta $}
		Let $X$ and $Y$ be topological spaces. A function $f : X \rightarrow Y$ is continuous if, for every $x \in X$ and every open set $U$ containing $f ( x )$ , there exists a neighborhood $V$ of $x$ such that $f ( V ) \subset U$ .\\
		\\
		$ \forall x\in X \text{ and every open set } U \text{ containing } f(x), \exists \text{ neighborhood } V \text{ of } x,$ such that $ f(V)\subset U $
	
	\subsection{Theorem that a function is continuous if and only if every element has a neighborhood containing $ f(x) $, there exists a neighbor$ V $ of $ x $ such that $ f(V)\subset U $}
		A function $f : X \rightarrow Y$ is continuous in the open set definition of continuity if and only if for every $x \in X$ and every open set $U$ containing $f ( x ) ,$ there exists a neighborhood $V$ of $x$ such that $f ( V ) \subset U$
		\begin{proofed}
			First, suppose that the open set definition holds for functions
			$f : X \rightarrow Y .$ Let $x \in X$ and an open set $U$ C $Y$ containing $f ( x )$ be
			given. Set $V = f ^ { - 1 } ( U ) .$ It follows that $x \in V$ and that $V$ is open in $X$
			since $f$ is continuous by the open set definition. Clearly $f ( V ) \subset U ,$ and
			therefore we have shown the desired result.\\
			\\
			Now assume that for every $x \in X$ and every open set $U$ containing
			$f ( x )$ , there exists a neighborhood $V$ of $x$ such that $f ( V ) \subset U .$ We show
			that $f ^ { - 1 } ( W )$ is open in $X$ for every open set $W$ in $Y$ . Thus let $W$ be an arbitrary open set in $Y .$ To show that $f ^ { - 1 } ( W )$ is open in $X ,$ choose an
			arbitrary $x \in f ^ { - 1 } ( W ) .$ It follows that $f ( x ) \in W ,$ and therefore there
			exists a neighborhood $V _ { x }$ of $x$ in $X$ such that $f \left( V _ { x } \right) \subset W ,$ equivalently,
			such that $V _ { x } \subset f ^ { - 1 } ( W ) .$ Thus, for an arbitrary $x \in f ^ { - 1 } ( W )$ there exists an open set $V _ { x }$ such that $x \in V _ { x } \subset f ^ { - 1 } ( W ) .$ Theorem 1.4 implies that
			$f ^ { - 1 } ( W )$ is open in $X .$
		\end{proofed}
	\subsection{Theorem that converges points will converge given a function}
		 Assume that $f : X \rightarrow Y$ is continuous. If a sequence
		$\left( x _ { 1 } , x _ { 2 } , \ldots \right)$ in $X$ converges to a point $x ,$ then the sequence $\left( f \left( x _ { 1 } \right) , f \left( x _ { 2 } \right) , \ldots \right)$
		in $Y$ converges to $f ( x )$ .
		\begin{proofed}
			Let $U$ be an arbitrary neighborhood of $f ( x )$ in $Y .$ Since $f$ is
			continuous, $f ^ { - 1 } ( U )$ is open in $X$ . Furthermore, $f ( x ) \in U$ implies that
			$x \in f ^ { - 1 } ( U ) .$ The sequence $\left( x _ { 1 } , x _ { 2 } , \ldots \right)$ converges to $x ;$ thus, there exists $N \in Z _ { + }$ such that $x _ { n } \in f ^ { - 1 } ( U )$ for all $n \geq N .$ It follows that
			$f \left( x _ { n } \right) \in U$ for all $n \geq N ,$ and therefore the sequence $\left( f \left( x _ { 1 } \right) , f \left( x _ { 2 } \right) , \ldots \right)$
			converges to $f ( x )$
		\end{proofed}
	\subsection{Theorem that we can map closed sets between each other}
		Let $X$ and $Y$ be topological spaces. A function $f : X \rightarrow Y$ is continuous if and only if $f ^ { - 1 } ( C )$ is closed in $X$ for every closed set $C \subset Y .$
		\begin{proofed}[My]
			Let $ C $ be a closed set in $ Y $. Notice, $ Y - C$ is open in $ Y $ and so $ f^{-1}(Y-C) $ must also be open in $ X $. \\
			We claim that $ f^{-1}(Y-C)=f^{-1}(Y)-f^{-1}(C) $ and is open.\\
			We define $ f^{-1}(Y-C) = \{x\in X| f(x)\in Y-C \} $. This implies that $ f(x)\in Y $ and $ f(x)\not \in C $. We also define $ f^{-1}(Y)=\{x\in X|f(x)\in Y \} $ and $ f^{-1}(C)=\{x\in X| f(x)\in C\} $ Thus, $ f^{-1}(Y)-f^{-1}(C) $ implies $ f(x)\in Y $ and $ f(x)\not\in C $. Notice, this is our definition of $f^{-1}(Y-C)$.\\
			Thus, $ f^{-1}(Y-C) \subseteq f^{-1}(Y)-f^{-1}(C)  $\\
			Going the other direction, we have the definition of $f^{-1}(Y)-f^{-1}(C) $ from our implication of $ f^{-1}(Y-C) $\\
			Thus, $ f^{-1}(Y)-f^{-1}(C) \subseteq f^{-1}(Y-C) $\\
			Therefore,  $ f^{-1}(Y-C)=f^{-1}(Y)-f^{-1}(C) $ \w{(Thank you for recognizing this needed to be proved.) }\\
			Since, $ f^{-1}(Y) $ is defined to be $ \{x\in X| f(x)\in Y\} $ we know that $ f^{-1}(Y)\subseteq X $. But since all of $ X $ is mapped in the preimage we also have $ X\subseteq f^{-1}(Y) $. Thus, $ f^{-1}(X)=Y $\w{(You mean $ f^{-1}(Y)=X $)}\\
			Taking $ X-C $ will result in an open set as $ X-C=X\cap C^\complement $ and since $ C^\complement $ is open and the intersection of open sets are open.\\
			Observe.
			\[f^{-1}(Y-C)=f^{-1}(Y)-f^{-1}(C)=X-f^{-1}(C)\]
			Thus, $ f^{-1}(C) $ must be closed by our previous result.\\
			\\
			Suppose, $ f^{-1}(C) $ is closed. We then know that $X-f^{-1}(C)$ must be open. Notice, that $ X=f^{-1}(Y) $. Observe. 
			\[X-f^{-1}(C) = f^{-1}(Y)-f^{-1}(C) = f^{-1}(Y-C) \]
			As, $ Y - C $ is open and $ f^{-1}$ is defined with an arbitrary open set $ U $ as $ f^{-1}(U)=\{x\in X| f(x)=U\} $, we then have that $ f^{-1} $ maps open sets to open sets.\\
			Thus, $ f $ must be continuous.\w{5/5 }
		\end{proofed}
	\subsection{Theorem that function composition works for continuity}
		Let $f : X \rightarrow Y$ and $g : Y \rightarrow Z$ be continuous. Then the
		composition function, $g \circ f : X \rightarrow Z ,$ is continuous.
		\begin{proofed}
			Suppose that $f : X \rightarrow Y$ and $g : Y \rightarrow Z$ are continuous,
			and let $U$ be an open set in $Z$ . Then $( g \circ f ) ^ { - 1 } ( U ) = f ^ { - 1 } \left( g ^ { - 1 } ( U ) \right)$ ,
			since $g$ is continuous, $g ^ { - 1 } ( U )$ is open in $Y$ , and since $f$ is continuous, $f ^ { - 1 } \left( g ^ { - 1 } ( U ) \right)$ is open in $X .$ Thus, $( g \circ f ) ^ { - 1 } ( U )$ is open in $X$ for an arbitrary $U$ open in $Z ,$ implying that $g \circ f$ is continuous.
		\end{proofed}
	\subsection{The Pasting Lemma}
		Let $ X $ be a topological space and let $ A $ and $B$ be closed subsets of $X$ such that $A \cup B = X .$ Assume that $f : A \rightarrow Y$ and $g : B \rightarrow Y$ are continuous and $f ( x ) = g ( x )$ for all $x$ in $A \cap B .$ Then $h : X \rightarrow Y ,$ defined by
			\[h ( x ) = \left\{ \begin{array} { l l } { f ( x ) \text { if } x \in A } \\ { g ( x ) \text { if } x \in B } \end{array} \right.\]
		is a continuous function.
		\begin{proofed}
			Proof. By Theorem 4.8 , it suffices to show that if $C$ is closed in $Y$ ,
			then $h ^ { - 1 } ( C )$ is closed in $X .$ Thus suppose that $C$ is closed in $Y .$ Note
			that $h ^ { - 1 } ( C ) = f ^ { - 1 } ( C ) \cup g ^ { - 1 } ( C ) .$ since $f$ is continuous, it follows by Theorem 4.8 that $f ^ { - 1 } ( C )$ is closed in $A$ . Theorem 3.4 then implies that
			$f ^ { - 1 } ( C ) = D \cap A$ where $D$ is closed in $X$ . Now, $D$ and $A$ are both closed
			in $X ,$ and $f ^ { - 1 } ( C ) = D \cap A ;$ therefore $, f ^ { - 1 } ( C )$ is closed in $X .$ Similarly, $g ^ { - 1 } ( C )$ is closed in $X .$ Thus, $h ^ { - 1 } ( C )$ is the union of two closed sets in
			$X$ and therefore is closed in $X$ as well. It follows that $h$ is continuous. 
		\end{proofed}
	\subsection{Definition of a Homeomorphism}
		Let $X$ and $Y$ be topological spaces, and let $f : X \rightarrow Y$ be a bijection with inverse $f ^ { - 1 } : Y \rightarrow X .$ If both $f$ and $f ^ { - 1 }$ are continuous functions, then $f$ is said to be a \textbf{homeomorphism}. If there exists a homeomorphism between $X$ and $Y ,$ we say that $X$ and $Y$ are \textbf{homeomorphic} or \textbf{topologically equivalent}, and we denote this by $X \cong Y .$
	\subsection{Facts about Homeomorphisms}
		\[\begin{array} { l } { \text { (i) The function } i d : X \rightarrow X , \text { defined by } i d ( x ) = x , \text { is a homeomorphism. } } \\
		 { \text { (ii) If } f : X \rightarrow Y \text { is a homeomorphism, then so is } f ^ { - 1 } : Y \rightarrow X \text { . } } \\
		  { \text { (iii) If } f : X \rightarrow Y \text { and } g : Y \rightarrow Z \text { are homeomorphisms, then so is }  g \circ f : X \rightarrow Z } \end{array}\]
		 \\
	\begin{defn}[4.15]{Embedding}
			An \textbf{embedding} of $X$ in $Y$ is a function $f : X \rightarrow Y$ that
			maps $X$ homeomorphically to the subspace $f ( X )$ in $Y .$
	\end{defn} 
	\begin{defn}[4.16]{Arc \& Simple Closed Curve}
			Let $X$ be a topological space. If $f : [ - 1,1 ] \rightarrow X$ is an
			embedding, then the image of $f$ is called an \textbf{arc} in $X$ , and if $f : S ^ { 1 } \rightarrow X$ is an embedding, then the image of $f$ is called a \textbf{simple closed curve} in $X$ .
	\end{defn}
	\begin{thm}[4.17]{Hausdorffness being a topological property}	  
		If $f : X \rightarrow Y$ is a homeomorphism and $X$ is Hausdorff,
		then $Y$ is Hausdorff.
		\begin{proofed}
			Suppose that $X$ is Hausdorff and $f : X \rightarrow Y$ is a homeomorphism. Let $x$ and $y$ be distinct points in $Y .$ Then $f ^ { - 1 } ( x )$ and $f ^ { - 1 } ( y )$ are distinct points in $X .$ Thus, there exist disjoint open sets $U$ and $V$ containing $f ^ { - 1 } ( x )$ and $f ^ { - 1 } ( y ) ,$ respectively. It follows that $f ( U )$ and
			$f ( V )$ are disjoint open sets containing $x$ and $y$ , respectively. Therefore
			$Y$ is Hausdorff.
		\end{proofed}
	\end{thm}
	
	\begin{defn}{Topological Property}
		A property of topological spaces that is preserved by homeomorphism is
		said to be a \textbf{topological property.}
	\end{defn}

	\section{Metric Spaces}
	\begin{defn}[5.1]{Metric}
		A \textbf{metric} on a set $X$ is a function  $d : X \times X \rightarrow \mathbb { R }$ with the following properties:
		\begin{enumerate}
			\item[(O)]  $d ( x , y ) = 0$ for some $x , y \in X ;$ if and only if $x = y$
			\item[(i)] $d ( x , y ) \geq 0$ for all $x , y \in X$
			\item[(ii)]  $d ( x , y ) = d ( y , x )$ for all $x , y \in X$
			\item[(iii)] $d ( x , y ) + d ( y , z ) \geq d ( x , z )$ for all $x , y , z \in X$
		\end{enumerate}
	\end{defn}
	
	\begin{defn}{Metric Space}
		We call $d ( x , y )$ the distance between $x$ and $y ,$ and we call the pair $( X , d )$ , consisting of the set $X$ and the metric $d$ , a \textbf{metric space}.
	\end{defn}

	\begin{defn}{Standard Metric}
		Given points $ p = {(p_1,p_2)} $ and $ q = (q_1,q_2) $
		\[d ( p , q ) = \sqrt { \left( p _ { 1 } - q _ { 1 } \right) ^ { 2 } + \left( p _ { 2 } - q _ { 2 } \right) ^ { 2 } }\]
		We call $d$ the\textbf{ standard metric} on $\mathbb { R } ^ { 2 }$ . This metric measures the straight-line distance between points in the plane.
	\end{defn}

	\begin{defn}{Taxicab Metric}
		Given points $ p = {(p_1,p_2)} $ and $ q = (q_1,q_2) $
		\[d _ { M } ( p , q ) = \max \left\{ \left| p _ { 1 } - q _ { 1 } \right| , \left| p _ { 2 } - q _ { 2 } \right| \right\}\]
		We call $d$ the\textbf{ max metric} on $\mathbb { R } ^ { 2 }$ . The distance in this metric is the maximum of the differences between their coordinates.
	\end{defn}
		  
	\begin{defn}{Max Metric}
		Given points $ p = {(p_1,p_2)} $ and $ q = (q_1,q_2) $
		\[d _ { M } ( p , q ) = \max \left\{ \left| p _ { 1 } - q _ { 1 } \right| , \left| p _ { 2 } - q _ { 2 } \right| \right\}\]
		We call $d$ the\textbf{ max metric} on $\mathbb { R } ^ { 2 }$ . This metric measures the distance between two points is the maximum of the differences between their coordinates.
	\end{defn}
	\begin{defn}[5.5]{Metric Topology}
		Let $( X , d )$ be a metric space. The topology generated by
		the basis of open balls $\B = \left\{ B _ { d } ( x , \varepsilon ) | x \in X , \varepsilon > 0 \right\}$ is called the topology
		induced by $d$ and is referred to as a\textbf{ metric topology.}
	\end{defn}	  
	\begin{thm}[5.6]{A set $ U $ is open iff for each $ y\in U $ there is an open ball centered at $ y $ and contained in $ U $}
		Let $( X , d )$ be a metric space. A set $U \subset X$ is open in the topology induced by $d$ if and only if for each $y \in U ,$ there is $a \delta > 0$ such that
		$B _ { d } ( y , \delta ) \subset U .$
	\end{thm}
\subsection*{Properties of Metric Spaces}
	\begin{thm}[5.12]{Every Metric Space is Hausdorff}
		\begin{proofed}
			Let $( X , d )$ be a metric space. Suppose $x$ and $y$ are distinct points
			in $X$ with $d ( x , y ) = \varepsilon .$ Consider the sets $U = B _ { d } ( x , \varepsilon / 2 )$ and $V =$
			$B _ { d } ( y , \varepsilon / 2 ) .$ It follows that $x \in U , y \in V ,$ and $U$ and $V$ are open sets. We claim that $U$ and $V$ are disjoint. Suppose $U \cap V \neq \varnothing ,$ and $z$ is in the
			intersection. Then $d ( x , z ) < \varepsilon / 2$ and $d ( y , z ) < \varepsilon / 2 .$ Therefore, by the
			triangle inequality,
			\[d ( x , y ) \leq d ( x , z ) + d ( z , y ) < \varepsilon / 2 + \varepsilon / 2 = \varepsilon\]
			that is, $d ( x , y ) < \varepsilon$ . This contradicts $d ( x , y ) = \varepsilon .$ Thus $U \cap V =$
			$\varnothing .$ Hence, there exist disjoint open sets $U$ and $V$ containing $x$ and $y$
			respectively, implying that $X$ is Hausdorff.
		\end{proofed}
	\end{thm}
	\begin{thm}[5.13]{}
		 Let $\left( X , d _ { X } \right)$ and $\left( Y , d _ { Y } \right)$ be metric spaces. A function
		$f : X \rightarrow Y$ is continuous in the open set definition if and only if for each $x \in X$ and $\varepsilon > 0 ,$ there exists $a \delta > 0$ such that if $x ^ { \prime } \in X$ and $d _ { X } \left( x , x ^ { \prime } \right) < \delta$
		then $d y \left( f ( x ) , f \left( x ^ { \prime } \right) \right) < \varepsilon$
		\begin{proofed}[My]
				(WTS: $\forall x\in X \exists \delta >0$ such that $ x'\in X $ and $ d_x(x,x')< \delta$, we have $ d_Y(f(x),f(x'))<\epsilon $)\\
			Suppose $ f $ is continuous. Let $ x\in X , \epsilon >0 $, and $ \delta > 0 $ such that $ x' \in X $ and $ d_x(x,x') < \delta $. Notice, as $ f $ is continuous, $ f(x), f(x')\in Y $. Since, $ x $ is bound by $\epsilon$, $ d(x,x') $ is bounded by $ \delta $, and $ f(x),f(x')\in Y $, we must have that $ d_Y(f(x),f(x') < \epsilon $.\\
			\\
			Let $ U $ be an open set in $ Y $. Let $ x\in f^{-1}(U) $ and define $ \epsilon > 0 $ such that $B(f(x),\epsilon)\subseteq U $. Define $ \delta $ such that$ x'\in X$ satisfies $ d(x,x')<\delta $. Which implies $ x'\in B(x,\delta) $. Notice, we must have $ d(f(x),f(x'))<\epsilon $. From this result, we have $ f(x')\in B(f(x),\epsilon)\subseteq U $. So, $ x'\in B(x,\delta) $ as $ x'\in f^{-1}(U) $.\\
			Thus, $ B(x,\delta) \subseteq f^{-1}(U) $.\\
			Thus, $ f $ is continuous.\\
			\\
			Therefore, A function
			$f : X \rightarrow Y$ is continuous in the open set definition if and only if for each $x \in X$ and $\varepsilon > 0 ,$ there exists a $\delta > 0$ such that if $x ^ { \prime } \in X$ and $d_X \left( x , x ^ { \prime } \right) < \delta$
			then $d _ { Y } \left( f ( x ) , f \left( x ^ { \prime } \right) \right) < \varepsilon .$
		\end{proofed}
	\end{thm}
	\begin{defn}[5.14]{Distance between}
		Let $( X , d )$ be a metric space. For sets $A , B \subset X$ define the distance between $A$ and $B$ by
			\[d ( A , B ) = g l b \{ d ( a , b ) | a \in A , b \in B \}\]
		The greatest lower bound in Definition 5.14 exists for every pair of sets $A$
		and $B$ since the set of values $\{ d ( a , b ) | a \in A , b \in B \}$ is bounded below by $0 .$
	\end{defn}
	\begin{thm}[5.15]{Proving one topology is finer than other with metrics}
		THEOREM $5.15 .$ Let $d$ and $d ^ { \prime }$ be metrics on a set $X ,$ and let $\mathcal { T }$ and $T ^ { \prime }$ ,
		respectively, be the topologies that they induce. Then $T'$ is finer than $\mathcal { T }$ if and only if for each $x \in X$ and $\varepsilon > 0 ,$ there exists $a \delta > 0$ such that $B _ { d ^ { \prime } } ( x , \delta ) \subset$
		$B _ { d } ( x , \varepsilon ) .$ 
		\begin{proofed}
			Proof. Suppose that $T ^ { \prime }$ is finer than $\mathcal { T }$ . Then every open set in $\mathcal { T }$ is open
			in $\mathcal { T } ^ { \prime }$ . In particular, for every $x \in X$ and $\varepsilon > 0 , B _ { d } ( x , \varepsilon )$ is open in $\mathcal { T }$ and hence is open in $T ^ { \prime } .$ since $B _ { d } ( x , \varepsilon )$ is open in $T ^ { \prime }$ and contains $x$
			Theorem 5.6 implies that there is a $\delta > 0$ such that $B _ { d ^ { \prime } } ( x , \delta ) \subset B _ { d } ( x , \varepsilon )$
			as we wished to show.\\
			\\
			Suppose now that for each $x \in X$ and $\varepsilon > 0 ,$ there exists a $\delta > 0$
			such that $B _ { d ^ { \prime } } ( x , \delta ) \subset B _ { d } ( x , \varepsilon ) .$ We prove that $T ^ { \prime }$ is finer than $\mathcal { T } .$ Let $U$
			be an open set in $\mathcal { T }$ . We show that $U$ is open in $\mathcal { T } ^ { \prime } .$ Let $x$ be an arbitrary point in $U$ . Since $U$ is open in $T ,$ Theorem 5.6 implies that there is an
			$\varepsilon > 0$ such that $B _ { d } ( x , \varepsilon ) \subset U .$ By assumption there is a $\delta > 0$ such that
			$B _ { d ^ { \prime } } ( x , \delta ) \subset B _ { d } ( x , \varepsilon ) \subset U .$ It follows that for each $x \in U$ there exists $\delta > 0$ such that $B _ { d^ { \prime } } ( x , \delta ) \subset U .$ Theorem 5.6 implies that $U$ is open in
			$\mathcal { T } ^ { \prime } ,$ as we wished to show.
		\end{proofed}
	\end{thm}
		\begin{defn}[5.17]{Bounded Metric}
			Let $( X , d )$ be a metric space. A subset $A$ of $X$ is said
			to be bounded under $d$ if there exists a $\mu > 0$ such that $d ( x , y ) \leq \mu$ for all $x , y \in A .$ If $X$ itself is bounded under $d ,$ then we say that $d$ is a \textbf{bounded
			metric.}
		\end{defn}
		\begin{thm}[5.18]{A topology is bounded if it is induced by a bounded metric}
			THEOREM 5.18. Let $( X , d )$ be a metric space, and define $d ^ { \prime } : X \times X \rightarrow \mathbb { R }$
			by $d ^ { \prime } ( x , y ) = \min [ d ( x , y ) , 1 \} .$ Then $d ^ { \prime }$ is a bounded metric that induces the
			same topology as $d$ .
		\end{thm}
		\begin{defn}[5.19]{Isometry}
			Let $\left( X , d _ { X } \right)$ and $\left( Y , d _ { Y } \right)$ be metric spaces. A bijective function $f : X \rightarrow Y$ is called an \textbf{isometry} if $d _ { X } \left( x , x ^ { \prime } \right) = d _ { Y } \left( f ( x ) , f \left( x ^ { \prime } \right) \right)$ for
			every pair of points $x$ and $x ^ { \prime }$ in $X$ . If $f : X \rightarrow Y$ is an isometry, then we say
			that the metric spaces $X$ and $Y$ are \textbf{isometric}.
		\end{defn}
	\section{Connectedness}
	\begin{defn}[6.1]{Connected}
		Let $ X $ be a topological space.\\
		\\
		(i) We call $X$ \textbf{connected} if there does not exist a pair of disjoint
		nonempty open sets whose union is $X .$\\
		\\
		(ii) We call $X$ \textbf{disconnected} if $X$ is not connected.\\
		\\
		(iii) If $X$ is disconnected, then a pair of disjoint nonempty open sets
		whose union is $X$ is called \textbf{a separation of $X .$}
	\end{defn}
	\begin{thm}[6.2]{A topological space $X$ is connected if and only if there are
			no nonempty proper subsets of $X$ that are both open and closed in $X$ .}
		\begin{proofed}[My]
			Suppose that $ X $ is not connected. Let $ U,V $ be a separation of $ X $. That is $ U\cup V = X$ and $ U,V $ are disjoint nonempty open sets. Notice, $ V=X-U  $ as $ U $ and $ V $ are disjoint. Then, we have that $ U $ is closed as it is the complement of an open set. We then have $ V=X-U\not= \varnothing $. This results says that $ U $ is a proper subset. Thus, $ U $ is a nonempty, proper, closed, and open set.\\
			\\
			Suppose $ U$ is a nonempty proper set that is closed and open. Let $ V=X-U $. Since, $ U $ is proper we have $ V $ is nonempty and open. Notice.
			\[X=U\cup(X-U)=U\cup V\]
			Thus, $ U,V $ form a separation of $ X $.\\
			Thus, $ X $ is not connected.\\
			\\
			Therefore, $X$ is connected if and only if there are no nonempty proper subsets of $X$ that are both open and closed in $X .$\\
		\end{proofed}
	\end{thm}
	\begin{thm}[6.4]{Showing disconnected}
		 A set $ A $ is disconnected in $X$ if and only if there exist open
		sets $U$ and $V$ in $X$ such that $A \subset U \cup V , U \cap A \neq \varnothing , V \cap A \neq \varnothing ,$ and
		$U \cap V \cap A = \varnothing .$
		\begin{proofed}
			Suppose that $A$ is disconnected in $X .$ Then there exist nonempty $y$ . Then there exist nonempty
			sets $P$ and $Q$ that are open in $A$ , disjoint, and such that $P \cup Q = A$ .
			since $P$ and $Q$ are open in $A$ there exist sets $U$ and $V$ that are open in $X$ and such that $U \cap A = P$ and $V \cap A = Q .$ Clearly, $A \subset U \cup V ,$
			$U \cap A \neq \varnothing , V \cap A \neq \varnothing ,$ and $U \cap V \cap A = \varnothing$ . \\
			\\
			Now suppose that $U$ and $V$ are open sets in $X$ such that $A \subset U \cup V ,$
			$U \cap A \neq \varnothing , V \cap A \neq \varnothing ,$ and $U \cap V \cap A = \varnothing .$ If we let $P = U \cap A$ and
			$Q = V \cap A ,$ then it follows that the pair of sets, $P$ and $Q ,$ is a separation
			of $A$ in the subspace topology, and therefore $A$ is disconnected in $X$ .
		\end{proofed}
	\end{thm}
	\begin{defn}[6.5]{Separation}
		Let $  A $ be a subspace of a topological space $ X $. If $U$ and
		$V$ are open sets in $X$ such that $A \subset U \cup V , U \cap A \neq \varnothing , V \cap A \neq \varnothing ,$ and $U \cap V \cap A = \varnothing ,$ then we say that the pair of sets, $U$ and $V ,$ is a separation
		of $A$ in $X .$
	\end{defn}
	\subsection*{There's a whole lot more that goes in here. I just don't have the time to be spending on this document.}
	\begin{defn}[6.27]{Path Connected}
		A topological space $X$ is \textbf{path connected} if for every $x , y \in X$ there is a path in $X$ from $x$ to $y . A$ subset of a topological space
		$X$ is path connected in $X$ is path connected in the subspace topology that
		A inherits from $X .$
	\end{defn}
	\begin{thm}[6.28]{If $ X $ is path connected space, than it is connected.}
		\begin{proofed}
			Proof. Let $X$ be a path connected space. We prove that $X$ is connected
			by showing that it has only one component, or equivalently that every pair
			of points $x , y \in X$ is contained in some connected subset of $X .$ Thus, let $x$ and $y$ be arbitrary points in $X .$ since $X$ is path connected, there is
			a path in $X$ from $x$ to $y .$ The image of such a path is a connected subset
			of $X$ containing both $x$ and $y .$ Therefore every pair of points in $X$ is
			contained in a connected subset of $X ,$ and it follows that $X$ is connected.
		\end{proofed}
	\end{thm}
	\\
	\section{Compactness 7.1}
	Let $A$ be a subset of a topological space $X ,$ and let $\mathcal { O }$ be a collection of subsets of $ X $
	\subsection{Definition of Cover}
	The collection $\mathcal { O }$ is said to \textbf{cover} $A$ if $A$ is
	contained in the union of the sets in $\mathcal { O }$ .
	\subsection{Definition of Open Cover}
	If $\mathcal { O }$ covers $A ,$ and each set in $\mathcal { O }$ is open, then we call $\mathcal { O }$ an \textbf{ open cover} of $A .$ 
	\subsection{Definition of Subcover}
	If $\mathcal { O }$ covers $A ,$ and $\mathcal { O } ^ { \prime }$ is a subcollection of $\mathcal { O }$ that also covers $A$
	then $\mathcal { O } ^ { \prime }$ is called a \textbf{subcover} of $\mathcal { O } .$\\
	
	\subsection{Definition of Compact}
	A topological space $X$ is \textbf{compact} if every open cover of $ X $ has a finite subcover.
	
	\subsection{Definition of Compact In}
	Let $X$ be a topological space, and assume $A \subset X .$ Then $A$ is said to be \textbf{compact in} $X$ if $A$ is compact in the subspace topology inherited from $X .$ 
	
	
	\subsection{Lemma: Checks weather or not a subspace A is compact}
	Let $X$ be a topological space, and assume $A \subset X .$ Then $ A $ is compact in $ X $ if and only if every cover of $A$ by sets that are open in $X$ has a finite subcover.\\
	\\
	Proof:\\
	Let $A$ be compact in $X ,$ and suppose that $\mathcal { O }$ is a cover of $A$ by
	open sets in $X .$ Then $\mathcal { O } ^ { \prime } = \{ U \cap A | U \in \mathcal { O } \}$ is a cover of $A$ by open sets in $A .$ Hence, there exists a finite subcover $\left\{ U _ { 1 } \cap A , U _ { 2 } \cap A , \ldots , U _ { n } \cap A \right\}$
	of $\mathcal { O } ^ { \prime } .$ But then $\left\{ U _ { 1 } , U _ { 2 } , \ldots , U _ { n } \right\}$ is a finite subcover of $\mathcal { O }$ . Therefore
	every cover of $A$ by open sets in $X$ has a finite subcover.\\
	\\
	Conversely, suppose every cover of $A$ by sets that are open in $X$
	has a finite subcover. Let $\mathcal { O } = \left\{ V _ { \beta } \right\} _ { \beta \in B }$ be a cover of $A$ by open sets
	in $A .$ Then, by definition of the subspace topology, for each $V _ { \beta }$ there vis an open set $U _ { \beta }$ in $X$ such that $V _ { \beta } = U _ { \beta } \cap A .$ It follows that the
	collection $\mathcal { O } ^ { \prime } = \left\{ U _ { B } \right\} _ { B \in R }$ is a cover of $A$ by open sets in $X$ . Since $\mathcal { O } ^ { \prime }$ has a finite subcover $\left\{ U _ { \beta _ { 1 } } , \ldots , U _ { \beta _ { n } } \right\} ,$ it follows that $\left\{ V _ { \beta _ { 1 } } , \ldots , V _ { \beta _ { n } } \right\}$ is a
	finite subcover of $\mathcal { O }$ . Thus every cover of $A$ by open sets in $A$ has a finite subcover, and therefore $ A $ is compact.\\
	\subsection{Compactness will be preserved through continuous functions}
	Let $f : X \rightarrow Y$ be continuous, and let $A$ be compact in $X .$ Then $ f(A) $ is compact in $ Y $.\\
	Proof:\\
	
	\subsection{Compact sets unioned together are compact}
	Let $ X $ be a topological space. If $C _ { 1 } , \ldots , C _ { n }$ are each compact in $X ,$ then $U _ { j = 1 } ^ { n } C _ { j }$ is compact	in $X .$
	\subsection{Intersection of Hausdorff compact sets are compact}
	If $X$ is Hausdorff, and $\left\{ C _ { \alpha } \right\} _ { \alpha \in A }$ is a collection of sets that are	compact in $X ,$ then $\bigcap _ { \alpha \in A } C _ { \alpha }$ is compact in $X$ .
	\subsection{If a subset of a compact set is closed, then that subset is also compact}
	Let $X$ be a topological space and let $D$ be compact in $X $. If $C$ is closed in $X ,$ and $C \subset D ,$ then $C$ is compact in $X$ .
	\subsection{All compact subsets of a Hausdorff space are closed}
	Let $X$ be a Hausdorff topological space and $A$ be compact in $X$ . Then $A$ is closed in $X .$
	\subsection{Tube Lemma i.e. You can take a slice of a space and it'll be compact still}
	Let $X$ and $Y$ be topological spaces, and assume that $Y$ is compact. If $x \in X ,$ and $U$ is an open set in $X \times Y$ containing $\{ x \} \times Y ,$ then there exists a neighborhood $W$ of $x$ in $X$ such that $W \times Y \subset U$.
	\subsection{Product topology preserves compactness}
	THEOREM $7.10 .$ If $X$ and $Y$ are compact topological spaces, then the product $X \times Y$ is compact.
	\setcounter{section}{6}
	\section{Compactness in Metric Spaces 7.2}
	\subsection{Closed Bounded Intervals are compact}
	Every closed and bounded interval $[ a , b ]$ is a compact subset of $\mathbb { R }$ with the standard topology.	
	\subsection{Product of closed bounded intervals are compact}
	Let $\left[ a _ { 1 } , b _ { 1 } \right] , \ldots , \left[ a _ { n } , b _ { n } \right]$ be closed bounded intervals in R. Then $\left[ a _ { 1 } , b _ { 1 } \right] \times \ldots \times \left[ a _ { n } , b _ { n } \right]$ is a compact subset of $\mathbb { R } ^ { n }$
	
	\subsection{The standard topology in standard metric in $ \R^n $ is compact iff it is closed and bounded}
	Let $\mathbb { R } ^ { n }$ have the standard topology and the standard metric
	d. A set $A \subset \mathbb { R } ^ { n }$ is compact in $\mathbb { R } ^ { n }$ if and only if it is closed and bounded.
	
	\subsection{In a metric space with compact subset $ A $, if there is a sequence, then there is a subsequence that converges to a limit in $ A $}
	Let $( X , d )$ be a metric space, and assume that $A$ is compact in $X .$ If $\left( x _ { n } \right)$ is a sequence in $A ,$ then there exists a subsequence $\left( x _ { n _ { m } } \right)$ of $\left( x _ { n } \right)$
	that converges to a limit in $A .$

\end{document}